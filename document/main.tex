\documentclass[12pt,a4paper]{report}

\usepackage[utf8]{inputenc}
\usepackage[english]{babel}
\usepackage{amsmath}
\usepackage{amsfonts}
\usepackage{amssymb}
\usepackage{graphicx}
\usepackage{hyperref}
\usepackage{todonotes}
\usepackage{xcolor}
\usepackage{listings}
\usepackage{authblk}
\usepackage{algorithm, algpseudocode}
\usepackage{geometry}
\usepackage{booktabs} 
\usepackage{longtable} 
\usepackage{array}
\usepackage{geometry}
\usepackage{tikz}
\usepackage{caption} 

\renewcommand{\lstlistingname}{Example}
\newcommand{\ie}{\textit{i.e.}}

%customized todo
\definecolor{lightgreen}{rgb}{0.8,1.0,0.8}
\definecolor{red}{rgb}{0.8,0,0}
\definecolor{green}{rgb}{0,0.8,0}
\definecolor{blue}{rgb}{0,0,1}
% Define some colors for listings
\definecolor{codegreen}{rgb}{0,0.6,0}
\definecolor{codegray}{rgb}{0.5,0.5,0.5}
\definecolor{codepurple}{rgb}{0.58,0,0.82}
\definecolor{backcolour}{rgb}{0.95,0.95,0.92}
\definecolor{keywordblue}{rgb}{0.13,0.13,1}
\definecolor{stringred}{rgb}{0.8,0,0}

% Listings style for BNFC commands
\lstdefinestyle{lbnf}{
    backgroundcolor=\color{backcolour},   
    commentstyle=\color{codegreen},
    keywordstyle=\color{keywordblue}\bfseries,
    numberstyle=\tiny\color{codegray},
    stringstyle=\color{stringred},
    basicstyle=\ttfamily\footnotesize,
    breakatwhitespace=false,         
    breaklines=true,                 
    captionpos=b,                    
    keepspaces=true,                 
    numbers=left,                    
    numbersep=5pt,                  
    showspaces=false,                
    showstringspaces=false,
    showtabs=false,                  
    tabsize=1,
    frame=lines,
    xleftmargin=2em,
    framexleftmargin=1.5em,
    escapeinside={\%*}{*} 
}

% Listings style for Bash/CLI commands
\lstdefinestyle{bash}{
    backgroundcolor=\color{backcolour},   
    commentstyle=\color{codegreen},
    keywordstyle=\color{keywordblue}\bfseries,
    numberstyle=\tiny\color{codegray},
    stringstyle=\color{stringred},
    basicstyle=\ttfamily\footnotesize,
    breakatwhitespace=false,         
    breaklines=true,                 
    captionpos=b,                    
    keepspaces=true,                 
    numbers=left,                    
    numbersep=5pt,                  
    showspaces=false,                
    showstringspaces=false,
    showtabs=false,                  
    tabsize=2,
    frame=lines,
    xleftmargin=2em,
    framexleftmargin=1.5em,
    escapeinside={\%*}{*} 
}

\newcommand{\mytodo}[1]{\todo[inline,color=lightgreen]{TODO:#1}}
\newcommand{\mnote}[2][]{\todo[inline,color=blue!10,#1]{Matthew: #2}}
\newcommand{\myremark}[1]{\todo[inline, color=lightgreen]{\textbf{Remark:} #1}}

\title{
	The VNN-LIB Standard \\ 
	Version 2.0 (draft)
}

\author[1]{Stefano Demarchi}
\author[2]{Dario Guidotti}
\author[2]{Luca Pulina}
\author[1]{Armando Tacchella}

\author[3]{Ann Roy}
\author[3]{Allen Antony}
\author[3]{Matthew Daggitt}

\affil[1]{University of Genoa, Viale Causa 13, 16145 Genoa, Italy}
\affil[2]{University of Sassari, Via Roma 151, 07100 Sassari, Italy}
\affil[3]{University of Western Australia, 35 Stirling Hwy, Crawley WA 6009, Australia}
  
\begin{document}

\maketitle

\begin{abstract}
This document presents VNN-LIB, a standard that formalises the query language for neural network verifiers. The standard uses the 
Open Neural Network Exchange (ONNX) format for model description and builds upon the Satisfiability Modulo Theories Library (SMT-LIB) 
format for query specification. Key among the standard is a formally defined syntax and semantics, complimentary tooling, as well as a 
command-line interface for verifiers. The goal is to foster greater robustness and interoperability in the neural network verification 
landscape.
\end{abstract}

\chapter{Introduction}
\label{sec:intro}

\section{Motivation}

While neural networks have shown exceptional performance across a range of tasks, 
they remain vulnerable to issues such as adversarial examples~\cite{szegedy2013intriguing} and discriminatory behaviour~\cite{4}. As these models are increasingly deployed in safety-critical and high-impact societal applications~\cite{1,2,3}, the need for robust methods to verify their behaviour has become paramount.

As with many formal verification problems, neural network verification can typically be reduced to sets of satisfiability queries. These queries are answerable by domain specific solvers which will be referred to as \emph{neural network verifiers} or simply \emph{verifiers} in this document. The \vnnlib{} standard was inspired by the success of SMT-LIB, which provides a unified format for queries to SMT solvers. 
Since its introduction in 2023, the \vnnlib{} standard~\cite{5} has been the de-facto specification language for queries in the neural network verification community, most notably used in the annual VNN-COMP~\cite{7} competition. Its goal is to facilitate the standardisation of solver interfaces and enable the collection of neural network verification benchmarks in a common format. 

\section{Document structure}

This document assumes some basic familiarity with neural networks, first-order logic and simple type theory. The document is orgnaised as follows:
\begin{enumerate}
\item \textbf{Chapter~\ref{sec:models} - \nameref{sec:models}}: A high-level description of the pre-existing ONNX standard for representing the neural network models.
\item \textbf{Chapter~\ref{sec:specification_language} - \nameref{sec:specification_language}}: The \vnnlib{} language for representing a satisfiability query over one or more neural network models.
\item \textbf{Chapter~\ref{sec:theories-logics} - \nameref{sec:theories-logics}}: A system for describing the different subsets of the query language that a given solver may support.
\item \textbf{Chapter~\ref{sec:solver_interface} - \nameref{sec:solver_interface}}: A standardised interface for allowing users to invoke a neural network verifier on a query and to query the capabilities of the verifier.
\end{enumerate}

\section{Document versioning}

New versions of the document are identified by semantic versioning. In general patch versions will fix bugs in the document, minor versions will include conservative extensions to the specification, while changes to the major version number will result from major backwards-incompatible changes.

\subsection*{Changelog for v2.0}

\paragraph{Authors:} Anthony Allen, Ann Roy, Matthew Daggitt with regular input from Stefano Demarchi, Andrea Gimelli.

\noindent \paragraph{Acknowledgements:} This version of the standard was developed with the help of many others in the neural network verification community.
Particular thanks are due to the following people, who gave many helpful suggestions, constructive criticism and encouragement: Taylor Johnson, Samuel Teuber, Wen Kokke, Julien Girard, Guilhem Ardouin, Augustin Lemesle, Michele Alberti, Julien Lehmann, Guy Katz, Thomas Flinkow, Edoardo Manino, Omri Isac, David Shriver, Christopher Brix.

\noindent \paragraph{Changes from previous version:}
\begin{itemize}
\item Formal grammar for query language
\item Added explicit network declarations to the query language.
\item Add support for networks with multiple inputs/outputs and hidden layers.
\item Added support for multiple networks
\item Added a formal type system and semantics.
\item Added theories and logics.
\item Introduction of the command-line interface: \texttt{verify} and \texttt{supports}.
\end{itemize}

\subsection*{Changelog for v1.0}

\paragraph{Authors:} Stefano Demarchi, Dario Guidotti, Luca Pulina, Armando Tacchella

\paragraph{Contents:}
\begin{itemize}
\item Initial release.
\item Outline of goal
\item Proposal of initial syntax of the query language.
\end{itemize}

\chapter{Neural Network Models}
\label{sec:models}
%
A neural network is a directed computation graph, where nodes correspond to operations (e.g., addition, multiplication, activation functions) over tensors, and edges represent the flow of data between these operations.
Inference is performed by applying these operations to one or more input tensors, propagating values through the graph, and producing one or more output tensors.

To formally reason about whether a query about a neural network is satisfiable, agreeing upon a standardised representation for neural network models is essential.
There are many such formats for modelling neural networks, often specialised to particular use cases. 
Formats such as TensorFlow’s \texttt{.ckpt}~\cite{Abadi_TensorFlow_Large_scale_machine_2015} or PyTorch’s \texttt{.pth}~\cite{Ansel_PyTorch_2_Faster_2024} are used within their own specific ecosystems, aimed at providing representations that facilitate the training of the model. 
In contrast, interoperable formats aim to decouple model representation from a particular framework or hardware backend, enabling broader reuse and tool support. 
Notable examples include ONNX (Open Neural Network Exchange)~\cite{onnxruntime}, NNEF (Neural Network Exchange Format)~\cite{nnef}, GGUF (GPT-Generated Unified Format)~\cite{gguf} and Safetensors~\cite{safetensors}.

\section{ONNX}
\label{sec:onnx}

The \vnnlib{} standard uses ONNX~\cite{onnxruntime} as the standard format for representing neural network models. This choice was motivated by the following reasons:
\begin{itemize}
\item \textbf{Community Owned}: ONNX is a community-driven rather than a proprietary project. This reduces the risk of the discontinuation of the standard and ensures greater neutrality and transparency in its governance.
\item \textbf{Framework and Hardware Agnostic}: ONNX is designed to be independent of both training frameworks and deployment environments. This makes it well-suited to serve as a common exchange format across diverse toolchains and platforms.
\item \textbf{Widespread Adoption}: ONNX is widely supported by major frameworks, including PyTorch and TensorFlow, which provide tooling for exporting to and importing from the ONNX format. This facilitates model sharing and conversion.
\item \textbf{Rich and Extensible Operator Set}: ONNX includes a comprehensive set of standardised operators, enabling it to express a wide variety of neural network architectures. Its extensibility also allows new operators to be proposed and adopted by the community as needed.

\item \textbf{Versioning and Documentation}: The ONNX specification includes detailed documentation and operator definitions, and a robust versioning system. This makes it easier for tools to implement and interact with the format in a consistent manner.

\item \textbf{Existing Solver Ecosystem}: Finally, ONNX is already well-integrated into the neural network verification research community. It has served as the standard format in the VNN-COMP competition, making it a natural fit for \vnnlib{} and aligning with existing solver tooling.
\end{itemize}
We will now describe ONNX's syntax (i.e. the format of ONNX models) and ONNX's semantics (i.e. the computation that ONNX models represent).

\subsection{Syntax}
\label{sec:onnx_overview}

The ONNX format provides a standardised syntax for specifying the computations performed by a neural network.
For a more detailed description suitable for implementators of neural network solvers, the full ONNX specification should be consulted. 

\paragraph{Model structure} 

An ONNX network is serialised as a single binary file with the \texttt{.onnx} file extension. Serialisation of \texttt{.onnx} files is performed using Protobuf (Protocol Buffers), a language and platform-neutral mechanism for serializing structured data. 
Key data within a \texttt{.onnx} file includes:
\begin{itemize}
	\item \textbf{Metadata}: This describes attributes of the model such as ONNX version, author, and description.
	\item \textbf{Initializers}: A list of constant tensors that are used by the model, such as weights and biases.
	\item \textbf{Inputs}: A list of descriptions of the one or more tensors that the model expects as input. This includes a name, the type of data stored in the tensor, and the shape of the tensor.
	\item \textbf{Nodes}: A list of nodes in the graph each representing a single operation on tensors (e.g., convolution, activation functions). Each node has a list of named input tensors and a list of named output tensors representing its connections in the graph.
	\item \textbf{Outputs}: A list of descriptions of the one or more tensors that the model produces as output. This includes a name, the type of data stored in the tensor, and the shape of the tensor.
\end{itemize}


\paragraph{Tensors}
All tensors processed by ONNX models are strongly typed. The basic properties of an ONNX tensor include:
\begin{itemize}
	\item \textbf{Element Type}: ONNX defines a \href{https://onnx.ai/onnx/repo-docs/IR.html#tensor-element-types}{standard set of types}. 
	The most commonly used types for neural network verification are: 32-bit floating point (float32), and 16-bit floating point (float16). Signed and unsigned integers (e.g., int32, uint8) are also supported, but are less commonly used.
	\item \textbf{Shape}: The shape of a tensor is defined as a list of integers, where each integer represents the size of the corresponding dimension. For example, a tensor with shape [3, 224, 224] 
	represents an image with 3 color channels (RGB) and dimensions 224\(\times\)224 pixels.
	\item \textbf{Values}: A contiguous block of memory containing the values for the elements of the tensor.
\end{itemize}

\paragraph{Operators and Opsets}

An ONNX \emph{operator} represents a mathematical operation which takes some number of input tensors and creates some number of output tensors. Each operator is defined to have the following meta-data: 
\begin{itemize}
	\item a name.
	\item a list of expected input tensors.
	\item a list of expected output tensors.
	\item a list of attributes that affect the operation performed (e.g. kernel size and stride for a convolution-based operator).
\end{itemize}
The full list of ONNX operators can be found in the ONNX specification, but common ones include:
\begin{itemize}
\item Gemm (General Matrix Multiplication)
\item Conv (Convolutional)
\item MaxPool (Maximum pooling)
\item ReLU (Rectified Linear Unit)
\item Sigmoid (Logistic Unit)
\item Softmax (Softmax Unit)
\end{itemize}
ONNX uses a versioning system called \emph{opsets} to manage the evolution of the list of supported operators. Each opset supports a given set of operators, and the semantics of existing operators may change between opset versions.
This ensures that the definition and behaviour of each operator can evolve over time without breaking existing models. The ONNX model file declares the opset version it uses in its metadata.

\subsection{Semantics}
\label{sec:onnx-semantics}

The ONNX standard provides a syntax for specifying a neural network mdoel. However, to formally define a satisfiability query over such a model, requires a precise, mathematical description of the semantics of the model, i.e. the computation that the model represents. 

Unfortunately, like most other mainstream neural network formats, ONNX has no formal mathematical semantics. 
Instead, the intended behavior of each operator is described informally in natural language by the ONNX documentation in varying levels of detail and clarity. 
As a result, the exact computation has been observed to differ across implementations or hardware platforms.

Encouragingly, progress is being made: since 2024, the \emph{ONNX Safety-Related Profile} working group has been developing a restricted subset of ONNX with formal semantics. 
As of 2025, work on this is currently ongoing.

\section{Network Theories}
\label{sec:network-theory}

Chapter~\ref{sec:specification_language} will define the syntax and semantics of the \vnnlib{} query language for describing satisfiability problems over ONNX models of neural networks.
The goal of a satisfiability query is to reason about the behaviour of the function that the ONNX model represents (i.e. its semantics).
Therefore, the syntax and semantics of a query will unavoidably depend on the syntax and semantics of the ONNX models.
This dependency has two immediate consequences for \vnnlib{}:
\begin{enumerate}
\item The current absence of a formal semantics for ONNX described in Section~\ref{sec:onnx-semantics}, means that it is not currently possible to concretely define the semantics of \vnnlib{}.
\item Even if there was a formal semantics for ONNX available, both the syntax and semantics of the ONNX standard will continue to evolve. 
It would be undesirable to tie the \vnnlib{} standard to a specific version of ONNX, as that would require new versions of the \vnnlib{} standard to be released in lockstep with each new version of the ONNX standard.
\end{enumerate}
The solution to these problems is define the syntax and semantics of \vnnlib{} \emph{relative} to some abstract theory of neural networks.
In practice, this means the \vnnlib{} query language will be parameterised by an abstract signature, $\networkTheoryVar$, containing the minimal set of syntax, typing judgments and semantics for neural networks necessary to define the syntax and semantics of the query language.
We will call the signature $\networkTheoryVar$ a \emph{network theory} . 

Therefore, although each new version of the ONNX standard will define a new network theory, complete with its own operators, element types and semantics, the \vnnlib{} standard will remain constant relative to an abstract network theory~$\networkTheoryVar$.
This allows the ONNX standard to evolve independently without requiring the explicit redefinition of the syntax and semantics of \vnnlib{}. 
One of the consequences of this parameterisation is that when discussing the semantics of a particular \vnnlib{} query, it is necessary to state the versions of both the ONNX standard and the \vnnlib{} standard being used.


\begin{figure}	
	\newcommand{\grammarShrink}{\vspace{-0.4em}}
	
	\begin{subfigure}{\textwidth}
		\setlength{\grammarindent}{9em}
	\begin{grammar}	
	<elementType> $\ni \elementTypeVar$ ::= \missing \qquad\qquad	<tensorName> $\ni \onnxNameVar$ ::= \missing
	
	<shape> $\ni \shapeVar$ ::= [$n_1$,...,$n_k$] \hspace{1em} for $n_i \in \mathbb{N}$ 
	
	<tensorType> $\ni \tensorTypeVar$ ::= <elementType>  $\times$ <shape>
	
	<modelType> $\ni \modelTypeVar$ ::= [<tensorType>]$^+$ $\times$ [<tensor-type>]$^+$
\end{grammar}
		\vspace{-0.7em}
		\caption{Abstract grammar for network model types.}
		\label{fig:onnx-type-syntax}
	\end{subfigure}
	\\
	\\
	\\
	\begin{subfigure}{\textwidth}
		\begin{minipage}[t]{0.44\textwidth}
	\begin{grammar}	
	<element> $\ni \elementVar$ ::= (-)[0-9]$^+$(.[0-9]$^+$)
	\end{grammar}
	\end{minipage}
	\hfill
	\begin{minipage}[t]{0.23\textwidth}
	\begin{grammar}	
	<tensor> $\ni \tensorVar$ ::= \missing
	\end{grammar}
	\end{minipage}
	\hfill
	\begin{minipage}[t]{0.24\textwidth}
	\begin{grammar}
	<model> $\ni \modelVar$ ::= \missing
	\end{grammar}
\end{minipage}
		\vspace{0.3em}
		\caption{Abstract grammar and functions for network models.}
		\label{fig:onnx-expr-syntax}
	\end{subfigure}
	\\
	\begin{subfigure}{\textwidth}
		% This minipage exists to force left-alignment
\begin{minipage}[t]{0.5\textwidth}
	\begin{flalign*}
		\inferrule*[Right=(Element)]{
            \missing
        }{
            \vdash \elementVar : \elementTypeVar
        }
        \hspace{8.3em}
        \inferrule*[Right=(Tensor)]{
            \missing 
        }{
            \vdash \tensorVar : \tensorTypeVar
        }
        \hspace{7.8em}
	    \inferrule*[Right=(Model)]{
            \missing
        }{
            \vdash \modelVar : \modelTypeVar 
        }
        \\
        \inferrule*[Right=(NodeOutput)]{
            \missing
        }{
            \modelVar \vdash \onnxNameVar : \tensorTypeVar
        }
        \hspace{9em}
        \inferrule*[Right=(Isomorphic)]{
            \missing
        }{
            \vdash \modelVar_1 \cong \modelVar_2
        }
        \hspace{5em}
    \end{flalign*}
\end{minipage}
		\caption{Abstract type system for network models.}
		\label{fig:onnx-types}
	\end{subfigure}
	\\
	\begin{subfigure}{\textwidth}
		
	% This minipage exists to force left-alignment
	\begin{minipage}[t]{0.5\textwidth}
	\setlength{\arraycolsep}{2pt}
	\begin{equation*}
	\begin{array}{llll}
	\semElementTypeAbs
	&: \mgrammar{<elementType>}
	&\rightarrow \set
	&= \missing 
	\\
	\semTensorAbs
	&: (\tensorVar : \tensorTypeVar )
	& \rightarrow \semTensorType{\tensorTypeVar}
	&= \missing 
	\\
	\semModelAbs
	&: (\modelVar : \modelTypeVar) 
	& \rightarrow \semModelType{\modelTypeVar}
	&= \missing
	\\
	\semModelHiddenAbs
	&: (\modelVar : \modelTypeVar) \rightarrow (\modelVar.\onnxNameVar : \tensorTypeVar) 
	& \rightarrow \semModelType{(\modelTypeVar.\text{inputs}, [\tensorTypeVar])} 
	&= \missing
	\end{array}
	\end{equation*}
	\vspace{-0.7em}
	\begin{equation*}
	\begin{array}{llll}
	\semTensorType{(\elementTypeVar, s)}
	&= \semElementType{\elementTypeVar}^{\prod_i s_i}
	\\
	\semModelType{(\tensorTypeVar^I, \tensorTypeVar^O)}
	&= \prod_{i} \semTensorType{\tensorTypeVar^I_i}  \rightarrow
	\prod_{i} \semTensorType{\tensorTypeVar^O_i}
	\end{array}
	\end{equation*}
	\begin{equation*}
	\vspace{0.5em}
	\begin{array}{llll}
	\compSem{\leq}
	\\
	\compSem{<}
	\\
	\compSem{\geq}
	\\
	\compSem{>}
	\\
	\compSem{=}
	\\
	\compSem{\neq}
	\\
	\opOneSem{-}
	\\
	\opTwoSem{+}
	\\
	\opTwoSem{\times}
	\end{array}
	\end{equation*}
\end{minipage}
		\caption{Abstract semantics for network models.}
		\label{fig:onnx-semantics}
	\end{subfigure}
	\caption{The definition of a network theory $\networkTheoryVar$, i.e. the minimal signature for an abstract implementation of ONNX that allows the syntax and semantics of VNN-LIB to be defined. The \missing{} symbol indicates what needs to be defined by the ONNX standard to instantiate the theory. Superscript $^+$ indicates a list of one or more.}
	\label{fig:onnx-signature}
\end{figure}


The definition of a network theory is shown in Figure~\ref{fig:onnx-signature}, with the missing components to be instantiated represented using the \missing{} symbol.
Figures~\ref{fig:onnx-type-syntax}~\&~\ref{fig:onnx-expr-syntax} describe the syntax of the types and expressions, with the interface requiring the definition of:
\begin{enumerate}
\item $\mgrammar{<elementType>}$ - a set of numeric element types
\begin{itemize}
\item In ONNX: types such as \texttt{float64}, \texttt{int32}.
\end{itemize}
\item $\mgrammar{<tensor>}$ - the format used to represent tensors 
\begin{itemize}
\item In ONNX: the \texttt{TensorProto} object.
\end{itemize}
\item $\mgrammar{<model>}$ - the format used to represent neural network models
\begin{itemize}
\item In ONNX: the \texttt{ModelProto} object.
\end{itemize}
\item $\mgrammar{<nodeOutput>}$ - the format used to reference the outputs of nodes
\begin{itemize}
	\item In ONNX: any string compliant with the C90 identifier syntax rules.
\end{itemize}
\item $\outputNodesFn$ - a function that maps a model to its list of its final outputs
\begin{itemize}
	\item In ONNX: the accessor \texttt{model.graph.output}.
\end{itemize}
\end{enumerate}
Figure~\ref{fig:onnx-types} describe the required typing judgments over the syntax, with the interface requiring the definition of:
\begin{enumerate}
\item \textsc{(Element)} - a judgement that a numeric string~$\elementVar$ is a valid element of a provided element type~$\elementTypeVar$. For example, it would be expected that the number~`1' could be judged as of type \texttt{float64} and \texttt{int32}, and `1.0' could be judged as of type \texttt{float64} but not \texttt{int32}.
\item \textsc{(Tensor)} - a judgement that a tensor~$\tensorVar$ is of a given element type and shape~$\tensorTypeVar$.
\item \textsc{(Model)} - a judgment that a model~$\modelVar$ is of a given type~$\modelTypeVar$, i.e takes in a list of tensors of the provided shape and produces a list of output tensors of the required shapes.
\item \textsc{(NodeOutput)} - a judgment that a model~$\modelVar$ has a node that produces some tensor called~$\onnxNameVar$ of type~$\tensorTypeVar$ as output.
\item \textsc{(Isomorphic)} - a judgment that the graph structure of two models are isomorphic to each other. See Section~\ref{sec:multiple-network} for details.
\end{enumerate}
Finally~\ref{fig:onnx-semantics} describe the semantics over the well-typed syntactic models, with the interface requiring the definition of:
\begin{enumerate}
\item $\semElementTypeAbs$ - a function mapping each syntactic element type to a mathematical set of values.
\item $\semTensorAbs$ - a function mapping a syntactic tensor $\tensorVar$ of type $\tensorTypeVar$ to the mathematical tensor that it represents.
\item $\semModelAbs$ - a function that takes a syntactic model $\modelVar$ that expects inputs of type~$\tensorTypeVar^I$ as its input and as its output returns another function. The returned function takes (i) a list of input tensors of type $\tensorTypeVar^I$ and (ii) an output of a node $\onnxNameVar$ of type $\tensorTypeVar$ and produces the value computed at that node output.
\item $\sem{\leq}$, ..., $\sem{\times}$ - the semantics of a list of basic operations that perform pointwise comparison and arithemetic over tensors.
\end{enumerate}
Note that nothing in the definition of a network theory is explicitly dependent on ONNX. 
The same mechanism could be used to define the syntax and semantics of the \vnnlib{} query language relative to other neural network modelling frameworks e.g. PyTorch. 
However, for the reasons outlined at the start of this chapter, in practice \vnnlib{} assumes the use of the ONNX format.
\chapter{\vnnlib{} Query Language}\label{sec:specification_language}

At the heart of the \vnnlib{} standard is the \vnnlib{} query language. Heavily influenced by SMT-LIB, this language is designed as a standardised computer-readable format for defining a wide range of satisfiability queries over neural networks. This chapter describes the syntax, scoping, typing and semantics of the query language.

\section{Syntax}
\label{sec:syntax}

The syntax of \vnnlib{} is formally defined as Labelled Backus-Naur Form~\cite{8} (LBNF) grammar which can be found in Appendix~\ref{app:lbnf_grammar}. Instead of going over the production rules in gory detail, we will now highlight key syntactic constructs of the language via examples illustrating their usage.
\vnnlib{} queries are split into two parts: network declarations and assertions.

\subsection{Network declarations}
\label{sec:network-declarations}
A network declaration is introduced by the keyword \texttt{declare-network}, followed by a user-defined variable name for the network, 
and then its associated input, hidden, and output variable declarations. Figure~\ref{fig:simple_net} shows a simple network declaration along with its ONNX
model representation.

\begin{figure}[h!]
    \begin{minipage}[c]{0.6\textwidth}
        \begin{lstlisting}[
            style=lbnf,
            label={lst:network_definition}
        ]
(declare-network simple_net
    (declare-input X Real [1,10])
    (declare-output Y Real [1,2])
)\end{lstlisting}
    \end{minipage}%
    \begin{minipage}[c]{0.45\textwidth}
        \centering
        \includegraphics[height=6cm]{imgs/simple_net.onnx.png}
    \end{minipage}
    \caption{A simple \vnnlib{} network declaration.}
    \label{fig:simple_net}
\end{figure}
\mnote{The diagrams look really good, but in all of them please can you:

1) make the networks match as closely as possible in height to the query next to them. For example in 3.1, can you remove one of the two hidden nodes? 

2) avoid naming the I/O nodes in the picture the same as the names used in the query. We don't want to give the impression that they need to be the same!

3) Render the input and output dimensions of the input and output nodes on the image. Those are really important as they do have to match. You can definitely see them on Netron.

4) Can you update all the queries to the syntax proposed in Issue \#63? }

All variables are declared inside of network declarations.  The network name (e.g., \texttt{simple\_net} below) is used by the verifier to 
associate the declared network with a specific ONNX file provided via the command line (as described in Chapter~\ref{sec:solver_interface}) while the variable names 
(e.g., \texttt{X}, \texttt{Y}) are used to reference nodes inside of the ONNX graph. 

All variable names follow the same syntax conventions. Variable names in \vnnlib{} are case-sensitive, must start with a letter, and may only contain letters, digits and underscores. All variable names must
be unique across the scope of the \vnnlib{} query. 

% The \texttt{@} character is a reserved character which is used to denote multiple applications of the same network, for the purpose of defining  hyperproperties such as monotonicity. For example \texttt{(declare-network acasXu@1 ...)} and \texttt{(declare-network acasXu@2 ...)} define two networks that are both instances of the same ONNX model,  denoted as \texttt{acasXu} in the command line interface of the verifier (See Chapter~\ref{sec:solver_interface} for more details).

\subsubsection*{Input and Output Variable Declarations}
\label{sec:input-output-declarations}
An input variable is declared using the \texttt{declare-input} keyword, followed by a variable name, its element type (e.g., \texttt{Real}, \texttt{int8}), 
and a list of integers representing the shape of the tensor. Similarly, an output variable uses the \texttt{declare-output} keyword. Multiple 
input and output variables can be declared within a single network declaration. There are two ways to map these declared variables to the nodes in the ONNX model:
\begin{enumerate}
    \item \textbf{Ordered Mapping (Default):} The variables are mapped to the ONNX graph's inputs/outputs based on their order of declaration. This is demonstrated in Example~\ref{lst:ordered_mapping}
    \item \textbf{Explicit Name Mapping:} Alternatively, variables can be explicitly mapped using its identifier within the ONNX graph. If this method is used, all input and output 
        variables within that network declaration must be given an explicit ONNX node name. This is demonstrated in Example~\ref{lst:named_mapping}.
\end{enumerate}

\begin{figure}[h!]
    \centering
    \begin{lstlisting}[
        caption={A network with multiple inputs/outputs, mapped by declaration order.},
        style=lbnf,
        label={lst:ordered_mapping}
    ]
(declare-network multi_io_net
    (declare-input image Real 1 3 224 224)
    (declare-input metadata Real 1 10)
    (declare-output logits Real 1 1000)
    (declare-output bbox Real 1 4)
)
    \end{lstlisting}
    \begin{lstlisting}[
        caption={The same network, but with explicit ONNX node name mapping.},
        style=lbnf,
        label={lst:named_mapping}
    ]
(declare-network multi_io_net
    (declare-input image Real 1 3 224 224 onnx-node:"image")
    (declare-input metadata Real 1 10 onnx-node:"metadata")
    (declare-output logits Real 1 1000 onnx-node:"logits")
    (declare-output bbox Real 1 4 onnx-node:"bbox")
)
    \end{lstlisting}
    \vspace{0.5cm}
    \includegraphics[height=10cm]{imgs/multi_io_net.onnx.png}
    \caption{A \vnnlib{} network declaration with multiple inputs and outputs. The first example uses ordered mapping, while the second uses explicit ONNX node names.}
    \label{fig:multi_io_net}
\end{figure}


\paragraph{Hidden Node Declarations}
\label{sec:hidden-node-declarations}
In some verification applications is necessary to reason about the result of intermediate computation results at hidden nodes within the network, such as encoding features, attention mechanisms, or other internal states.
This can be achieved by declaring hidden nodes is declared using the \texttt{declare-hidden} keyword. This declaration includes a variable name for use within the \vnnlib{} specification, 
its element type, its tensor shape, and crucially, a string identifier that specifies the corresponding node name in the ONNX graph.  Multiple
hidden nodes can be trivially declared within a single network declaration. Figure~\ref{fig:hidden_node} shows an example of a \vnnlib{} network declaration with a hidden node.

\begin{figure}[h!]
    \begin{minipage}[c]{0.55\textwidth}
        \begin{lstlisting}[
            style=lbnf,
            label={lst:hidden_node}
        ]
(declare-network encoder
    (declare-input X Real 1 28 28)
    (declare-hidden feature_embedding Real 1 128 onnx-node:"encoder_layer4/output")
    (declare-output Y Real 1 10)
)
        \end{lstlisting}
    \end{minipage}%
    \begin{minipage}[c]{0.45\textwidth}
        \centering
        \includegraphics[height=8cm]{imgs/encoder_net.onnx.png}
    \end{minipage}
    \caption{A \vnnlib{} network declaration with a hidden node. The hidden node \texttt{feature\_embedding} corresponds to the ONNX node \texttt{encoder\_layer4/output}.}
    \label{fig:hidden_node}
\end{figure}

\subsubsection*{Multiple networks}
\label{sec:multi-network-declarations}
\vnnlib{} supports defining multiple networks in a single file by including multiple `(declare-network ...)` expressions. This is essential for properties that compare networks, 
such as checking for equivalence between two models or verifying properties of a composite system, like an observer-controller architecture. Figure~\ref{fig:multi_network} 
shows an example of a \vnnlib{} file that declares two networks, \texttt{teacher\_net} and \texttt{student\_net}.

\begin{figure}[h!]
    \begin{minipage}[c]{0.57\textwidth}
        \begin{lstlisting}[style=lbnf]
(declare-network teacher
    (declare-input  x Real [1,32])
    (declare-output y Real [1,2])
)

(declare-network student
    (declare-input  a Real [1,32])
    (declare-output b Real [1,2])
)\end{lstlisting}
    \end{minipage}
    \begin{minipage}[c]{0.43\textwidth}
        \centering
        \includegraphics[height=8cm]{imgs/teacher_net.onnx.png}
        \vspace{0.5cm} 
        \includegraphics[height=8cm]{imgs/student_net.onnx.png}
    \end{minipage}
    \caption{Two networks declared in \vnnlib{}: \texttt{teacher\_net} and \texttt{student\_net}.}
    \label{fig:multi_network}
\end{figure}

\subsection{Assertion Example}

\subsubsection*{Assertion declarations}
\label{sec:assertion-declarations}
\vnnlib{} supports quantifier-free logical formulas as \textit{assertions}. Assertions are defined using parenthesized \texttt{(assert\ldots)} expressions, and following an SMT-LIB-like syntax with the 
operator preceding its operands. An assertion is a logical formula that may include logical connectives, relational comparisons, and arithmetic expressions over declared tensors and constants.

\subsubsection*{Variables and Indexing}
\label{sec:variables-and-indexing}
Assertions may only refer to individual elements of declared tensors. To refer to a specific scalar element within a tensor, an indexing notation is used. Let $X \in I$ be an $n$-dimensional tensor 
in some generic input domain $I = I^{d_1 \times \cdots \times d_n}$. The ``matrix notation'' represents a specific element $x_{i_1, i_2, \dots, i_n}$ of the tensor $X$ as \texttt{X[$i_1$,$i_2$,\dots,$i_n$]}, 
where $i_1, \dots, i_n$ are zero-based indices ranging from $0$ to $d_1{-}1$, $0$ to $d_2{-}1$, \dots, $d_n{-}1$, respectively. To better clarify, if we consider the 1-D tensor $X \in I^n$, the 2-D tensor 
$Y \in I^{n \times m}$, and the 3-D tensor $Z \in I^{n \times m \times p}$, we will have the following representations:
\begin{itemize}
    \item \texttt{X[0]}, \texttt{X[1]}, \dots, \texttt{X[$i$]}, \dots, \texttt{X[$n$]};
    \item \texttt{Y[0,0]}, \texttt{Y[0,1]}, \dots, \texttt{Y[$i$,$j$]}, \dots, \texttt{Y[$n$,$m$]};
    \item \texttt{Z[0,0,0]}, \texttt{Z[0,0,1]}, \dots, \texttt{Z[$i$,$j$,$k$]}, \dots, \texttt{Z[$n$,$m$,$p$]};
\end{itemize}
In such a representation, \texttt{Z[$i$-$j$-$k$]} corresponds to the element $z_{i,j,k}$ of the tensor $Z$. 

\subsubsection*{Arithmetic expressions}
Arithmetic expressions are formed using prefix notation with the following supported operators:
\begin{itemize}
    \item \texttt{(+ a b ...)}: Addition of two or more terms. 
    \item \texttt{(- a b ...)}: Subtraction of two or more terms. Alternatively, \texttt{(- a)} for negation.
    \item \texttt{(* a b ...)}: Multiplication of two or more terms. 
\end{itemize}
Operands (\texttt{a}, \texttt{b},...) can be constants, indexed tensors, or other nested arithmetic expressions.

\subsubsection*{Boolean expressions}
Boolean expressions are defined as expressions that produce a Boolean (\texttt{true} or \texttt{false}) value. They are formed using comparison operators and logical connectives:
\begin{itemize}
    \item \textbf{Comparison Operators:} \texttt{<=}, \texttt{>=}, \texttt{<}, \texttt{>}, \texttt{=}, \texttt{!=}
    \begin{itemize}
        \item The operands may be constants, indexed tensors, or arithmetic expressions. Each operator has two operands.
        \item For example \texttt{(<= a b)} returns true if $a$ is less than or equal to $b$.
    \end{itemize}
    \item \textbf{Logical Connectives:} \texttt{and}, \texttt{or}
    \begin{itemize}
        \item The operands must be Boolean expressions. Each operator can take two or more operands.
        \item For example \texttt{(and a b ...)} returns true if all operands are true.
    \end{itemize}
\end{itemize}

\begin{lstlisting}[
    caption={An assertion stating that if the input $A_0$ is between 0 and 1, the output $B_0$ must be greater than $B_1$ and their sum must be non-negative.},
    style=lbnf,
    label={lst:assertion-example}
]
(assert
    (and
        (and (>= A_0 0.0) (<= A_0 1.0))
        (and (> B_0 B_1) (>= (+ B_0 B_1) 0.0))
    )
)
\end{lstlisting}

\subsection{Comments and whitespace}

Comments in \vnnlib{} are denoted by a semicolon (\texttt{;}) and extend to the end of the line. They are used for annotation, explaining logic, or providing additional context. Whitespace in \vnnlib{} is used to separate tokens and improve readability. It can include spaces, tabs, and newlines. Whitespace is ignored by the parser, except where it is necessary to separate tokens.


\section{Scoping}
\label{sec:scoping}

Variable names must be unique within the scope of the entire \vnnlib{} specification.

TODO Ann

\section{Typing}
\label{sec:typing}

TODO Ann

\section{Semantics}
\label{sec:semantics}

TODO Ann



\chapter{Logics}
\label{sec:query_categories}

\mnote{I've been looking at the SMT-LIB standard (a really useful document! You should definitely have a browse!) and what these really are \emph{logics}: https://smt-lib.org/logics.shtml

We should have a similar, much smaller tree, but should cover all possible combinations of these.}

\section{Linearity}\label{sec:linearity}

\section{Reachability}\label{sec:reachability}

Inputs and outputs of operators are \emph{tensors}, i.e.,
multidimensional arrays over some domain, usually numerical. 
If we let $\mathbb{D}$ be any such domain, a $k$-dimensional 
tensor on $\mathbb{D}$ is denoted as $x \in \mathbb{D}^{n_1 
	\times \ldots \times n_k}$.
For example, a vector of $n$ real numbers is a 1-dimensional
tensor $x \in \mathbb{R}^n$, whereas a matrix of $n \times n$ 
Booleans is a 2-dimensional tensor $x \in \mathbb{B}^{n 
	\times n}$ with $\mathbb{B} = \{0, 1\}$. A specific element 
of a tensor can be singled-out via \emph{subscripting}. 

Given a $k$-dimensional tensor $x \in \mathbb{D}^{n_1 \times 
	\ldots \times n_k}$, the element $x_{i_1, \ldots, i_k} \in 
	\mathbb{D}$ is a scalar corresponding to the indexes 
${i_1, \ldots, i_k}$. For example, in a vector of real numbers 
$x \in \mathbb{R}^n$, $x_1$ is the first element, $x_2$ the second 
and so on. In a matrix of Boleans $x \in \mathbb{B}^{n \times
  n}$, $x_{1,1}$ is the first element of the first row, $x_{2,1}$ 
is the first element of the second and so on.

An \emph{operator} $f$ is a function on tensors 
$f: \mathbb{D}^{n_{1} \times n_h} \to \mathbb{D}^{m_{1} \times m_k}$
where $h$ is the dimension of the input tensor and $k$ is the 
dimension of the output tensor. Given a set $F = \{f_1, \ldots, 
	f_p\}$ of $p$ operators, a \emph{feedforward neural network}
is a function $\nu = f_p(f_{p-1}(\ldots f_2(f_1(x))\ldots))$ obtained
through the composition of the operators in $F$ assuming that the 
dimensions of their inputs and outputs are \emph{compatible}, i.e.,
if the  output of $f_i$ is a $k$-dimensional tensor, then the input
of $f_{i+1}$ is also a $k$-dimensional tensor, for all $1 \leq i < p$.

Given a neural network $\nu : \mathbb{D}^{n_{1} \times n_h} \to
\mathbb{D}^{m_{1} \times m_k}$ built on the set of operators $\{f_1,
\ldots, f_p\}$, let $x \in \mathbb{D}^{n_{1} \times n_h}$ denote
the input of $\nu$ and $y_1, \ldots, y_p$ denote the outputs of the
operators $f_1, \ldots, f_p$ --- therefore $y_p$ is also the output
$y$ of $\nu$. We assume that, in general, a \emph{property} is a first
order formula $P(x, y_1, \ldots y_p)$ which should be satisfied given 
$\nu$. More formally, given $p$ bounded sets $X_1, \ldots, X_p$ in $I$ 
such that $\Pi = \bigcup_{i=1}^p X_i$ and $s$ bounded sets $Y_1, 
\ldots, Y_s$ in $O$ such that $\Sigma = \bigcup_{i=1}^s Y_i$, we wish
to prove that  
\begin{equation}
	\label{eq:verif}
	\forall x \in \Pi \rightarrow \nu(x) \in \Sigma.
\end{equation}
The definition of the property given in equation (\ref{eq:verif})
consists of a \textit{pre-}condition $x \in \Pi$ and a 
\textit{post-}condition $\nu(x) \in \Sigma$. The 
\textit{pre-}condition encodes the bounds of the input space, i.e.,
bounds the variables that are fed to the network, and the 
\textit{post-}condition defines the safe zone, outside which the 
verification task fails.

\section{Disjunctive Normal Form}\label{sec:dnf}


% Define some colors for listings
\definecolor{codegreen}{rgb}{0,0.6,0}
\definecolor{codegray}{rgb}{0.5,0.5,0.5}
\definecolor{codepurple}{rgb}{0.58,0,0.82}
\definecolor{backcolour}{rgb}{0.95,0.95,0.92}
\definecolor{keywordblue}{rgb}{0.13,0.13,1}
\definecolor{stringred}{rgb}{0.8,0,0}

% Listings style for Bash/CLI commands
\lstdefinestyle{bash}{
    backgroundcolor=\color{backcolour},   
    commentstyle=\color{codegreen},
    keywordstyle=\color{keywordblue}\bfseries,
    numberstyle=\tiny\color{codegray},
    stringstyle=\color{stringred},
    basicstyle=\ttfamily\footnotesize,
    breakatwhitespace=false,         
    breaklines=true,                 
    captionpos=b,                    
    keepspaces=true,                 
    numbers=left,                    
    numbersep=5pt,                  
    showspaces=false,                
    showstringspaces=false,
    showtabs=false,                  
    tabsize=2,
    frame=lines,
    xleftmargin=2em,
    framexleftmargin=1.5em,
    escapeinside={\%*}{*} 
}

\section{Introduction}
This chapter specifies the command-line interface (CLI) for neural network verifiers compliant with the VNNLIB-2.0 standard. The goal is to provide a consistent and 
predictable interface for users and automated tools to interact with different verifiers. The CLI supports querying verifier capabilities, listing supported operations, 
and invoking the verification process with various configurations.

All verifiers adhering to this specification should expose an executable, referred to in this document as \texttt{vnn\_verifier}.

\section{General Usage}
The general syntax for interacting with the verifier via the CLI is:
\begin{lstlisting}[style=bash, numbers=none, frame=none, backgroundcolor=\color{white}]
vnn_verifier [global-options] <command> [command-options] 
	[arguments] \end{lstlisting}

\section{Global Options}
These options are applicable to the main \texttt{vnn\_verifier} executable and potentially to all commands.
\begin{itemize}
    \item \texttt{--help}, \texttt{-h}: Display help information for the CLI or for a specific command if used like \texttt{vnn\_verifier <command> --help}.
    \item \texttt{--name}, \texttt{-v}: Display the verifier's name.
    \item \texttt{--version}, \texttt{-V}: Display the verifier's version.
\end{itemize}

\section{Commands}

\subsection{\texttt{verify}}
\begin{itemize}
    \item \textbf{Description}: Invokes the verification process on a given neural network model and a VNNLIB property specification.
    \item \textbf{Usage}:
    \begin{lstlisting}[style=bash]
vnn_verifier verify
	 --network <name1>:<net1_path> 
	[--network <name2>:<net2_path> ...] 
	 --property <vnnlib_filepath> [options] \end{lstlisting}
    \item \textbf{Options}:
    \begin{longtable}{@{}>{\raggedright\arraybackslash\ttfamily}p{0.28\textwidth}% Option column
                    >{\raggedright\arraybackslash\ttfamily}p{0.12\textwidth}% Shorthand column
                    >{\raggedright\arraybackslash\ttfamily}p{0.20\textwidth}% Argument column
                    >{\raggedright\arraybackslash}p{0.32\textwidth}@{}}% Description column (normal font for readability)
        \toprule
        Option & Shorthand & Argument & Description \\
        \midrule
        \bottomrule
        \texttt{--network} & \texttt{-n} & \texttt{<name>:<path>} & \textbf{(Required)} Identifier for the network and path to the ONNX model file. \\
        \texttt{--property} & \texttt{-p} & \texttt{<path>} & \textbf{(Required)} Path to the VNNLIB 2.0 property file. \\
        \texttt{--timeout} & \texttt{-t} & \texttt{<seconds>} & Maximum verification time in integer seconds. \\
        \texttt{--dataset} & \texttt{-d} & \texttt{<path>} & Path to a dataset file (optional, for models that require input data).\\
        \texttt{--output} & \texttt{-o} & \texttt{<path>} & Path to save the satisfying assignment (if applicable). \\
    \end{longtable}
    \item \textbf{Standard Output/Error}:
    \begin{itemize}
        \item The primary result is printed to \texttt{stdout}: \texttt{SAT}, \texttt{UNSAT}, \texttt{TIMEOUT}, \texttt{UNKNOWN}.
        \item If \texttt{SAT}, the generated satisfying assignment will be printed to \texttt{stdout} or saved to the specified file.
        \item Detailed logs or error messages are sent to \texttt{stderr}.
    \end{itemize}
\end{itemize}

\myremark{The \texttt{<name>} in the \texttt{--network} option is a user-defined identifier for the network, which is used in the VNNLIB property file to refer to this specific network.}

\section{Granular Capability Options}
This section details additional global options for querying individual verifier capabilities. These options provide a convenient way to check for specific features of the verifier.

\subsection{\texttt{supports-multiple-networks}}
\begin{itemize}
    \item \textbf{Shorthand}: \texttt{mn}
    \item \textbf{Description}: Checks if the verifier supports properties defined over multiple neural networks. 
    \item \textbf{Usage}: \texttt{vnn\_verifier --supports-multiple-networks}
    \item \textbf{Output}: Prints "true" or "false" to stdout.
\end{itemize}

\subsection{\texttt{supports-multiple-input-output}}
\begin{itemize}
    \item \textbf{Shorthand}: \texttt{mio}
    \item \textbf{Description}: Checks if the verifier supports models with multiple input and/or output tensors. 
    \item \textbf{Usage}: \texttt{vnn\_verifier --supports-multiple-io}
    \item \textbf{Output}: Prints "true" or "false" to stdout.
\end{itemize}

\subsection{\texttt{supports-hidden-nodes}}
\begin{itemize}
    \item \textbf{Shorthand}: \texttt{hn}
    \item \textbf{Description}: Checks if the verifier supports referencing hidden layer nodes in VNNLIB properties. 
    \item \textbf{Usage}: \texttt{vnn\_verifier --supports-hidden-nodes}
    \item \textbf{Output}: Prints "true" or "false" to stdout.
\end{itemize}

\subsection{\texttt{requires-dnf}}
\begin{itemize}
    \item \textbf{Shorthand}: \texttt{dnf}
    \item \textbf{Description}: Checks if the verifier requires assertions to be in Disjunctive Normal Form (DNF). 
    \item \textbf{Usage}: \texttt{vnn\_verifier --requires-dnf}
    \item \textbf{Output}: Prints "true" or "false" to stdout.
\end{itemize}

\subsection{\texttt{supports-strict-comparison}}
\begin{itemize}
    \item \textbf{Shorthand}: \texttt{sc}
    \item \textbf{Description}: Checks if the verifier supports strict inequalities (\texttt{<}, \texttt{>}). 
    \item \textbf{Usage}: \texttt{vnn\_verifier --supports-strict-comparison}
    \item \textbf{Output}: Prints "true" or "false" to stdout.
\end{itemize}

\subsection{\texttt{requires-linear-complexity}}
\begin{itemize}
    \item \textbf{Shorthand}: \texttt{lc}
    \item \textbf{Description}: Checks if the verifier's supported assertions are limited to linear arithmetic. 
    \item \textbf{Usage}: \texttt{vnn\_verifier --requires-linear-complexity}
    \item \textbf{Output}: Prints "true" or "false" to stdout.
\end{itemize}

\subsection{\texttt{is-reachability-based}}
\begin{itemize}
    \item \textbf{Shorthand}: \texttt{rb}
    \item \textbf{Description}: Checks if the verifier is a reachability-based tool. 
    \item \textbf{Usage}: \texttt{vnn\_verifier --is-reachability-based}
    \item \textbf{Output}: Prints "true" or "false" to stdout.
\end{itemize}

\subsection{\texttt{supported-domains}}
\begin{itemize}
    \item \textbf{Shorthand}: \texttt{sd}
    \item \textbf{Description}: Prints a newline-separated list of the abstract input domain representations supported by the verifier. 
    \item \textbf{Usage}: \texttt{vnn\_verifier --list-input-domains}
    \item \textbf{Example Output}:
    \begin{lstlisting}[style=bash, numbers=none, frame=none, backgroundcolor=\color{white}]
Box
Zonotope
Polytope
...
    \end{lstlisting}
\end{itemize}

\subsection{\texttt{supported-types}}
\begin{itemize}
    \item \textbf{Shorthand}: \texttt{st}
    \item \textbf{Description}: Prints a newline-separated list of supported ONNX data types.
    \item \textbf{Usage}: \texttt{vnn\_verifier --list-supported-types}
    \item \textbf{Example Output}:
    \begin{lstlisting}[style=bash, numbers=none, frame=none, backgroundcolor=\color{white}]
float16
float32
...
    \end{lstlisting}
\end{itemize}

\subsection{\texttt{supported-onnx-ops}}
\begin{itemize}
    \item \textbf{Shorthand}: \texttt{so}
    \item \textbf{Description}: Prints a newline-separated list of ONNX operator types (e.g., ``Conv'', ``Relu'', ``Gemm'') that are supported by the verifier. 
	This helps users determine if their ONNX model is compatible with the verifier.
    \item \textbf{Usage}: \texttt{vnn\_verifier --list-onnx-ops}
    \item \textbf{Example Output}:
    \begin{lstlisting}[style=bash, numbers=none, frame=none, backgroundcolor=\color{white}]
Conv
Relu
MatMul
Gemm
Add
Flatten
...
    \end{lstlisting}
\end{itemize}

\appendix
\chapter{VNN-LIB LBNF Grammar}\label{app:lbnf_grammar}

\begin{thebibliography}{9} 

\bibitem{1} 
D. Tang, B. Qin, and T. Liu, ``Document modelling with gated recurrent neural network for sentiment classification,'' in \emph{Proceedings of the 2015 Conference on Empirical Methods in Natural Language Processing}, 2015, pp. 1422--1432.

\bibitem{2} 
M. Bojarski, et al., ``End to end learning for self-driving cars,'' \emph{arXiv preprint arXiv:1604.07316}, 2016.

\bibitem{3} 
C. Szegedy, et al., ``Intriguing properties of neural networks,'' \emph{arXiv preprint arXiv:1312.6199}, 2013.

\bibitem{4} 
P. Zhang et al., ``White-box fairness testing through adversarial sampling,'' in \emph{2020 IEEE/ACM 42nd International Conference on Software Engineering (ICSE)}, New York, NY, USA:\@ ACM, 2020, pp. 949--960.\'doi: \href{https://doi.org/10.1145/3377811.3380331}{10.1145/3377811.3380331}.

\bibitem{5} 
S. Demarchi, D. Guidotti, L. Pulina, and A. Tacchella, ``Supporting Standardization of Neural Networks Verification with VNN-LIB and CoCoNet,'' in \emph{Proc. 6th Int. Workshop on Formal Methods for ML-Enabled Autonomous Systems (FoMLAS 2023)}, 2023, pp. 47--58.

\bibitem{6} 
C. Brix, S. Bak, C. Liu, and T. T. Johnson, ``The Fourth International Verification of Neural Networks Competition (VNN-COMP 2023): Summary and Results,'' 2023, doi: \href{https://doi.org/10.48550/arxiv.2312.16760}{10.48550/arxiv.2312.16760}.

\bibitem{7} 
L. C. Cordeiro et al., ``Neural Network Verification is a Programming Language Challenge,'' 2025, doi: \href{https://doi.org/10.48550/arxiv.2501.05867}{10.48550/arxiv.2501.05867}.

\bibitem{8} 
M. Forsberg and A. Ranta, ``The Labelled BNF Grammar Formalism,'' Department of Computing Science, Chalmers University of Technology and the University of Gothenburg, Gothenburg, Sweden, Feb. 11, 2005. [Online]. Available: \url{https://bnfc.digitalgrammars.com/LBNF-report.pdf}

\end{thebibliography}

\end{document}
