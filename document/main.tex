\documentclass[12pt,a4paper]{report}

\usepackage[utf8]{inputenc}
\usepackage[english]{babel}
\usepackage{amsmath}
\usepackage{amsfonts}
\usepackage{amssymb}
\usepackage{graphicx}
\usepackage{hyperref}
\usepackage{todonotes}
\usepackage{xcolor}
\usepackage{listings}
\usepackage{authblk}
\usepackage{algorithm, algpseudocode}
\usepackage{geometry}
\usepackage{booktabs} 
\usepackage{longtable} 
\usepackage{array}


\newcommand{\ie}{\textit{i.e.}}
%customized todo
\definecolor{lightgreen}{rgb}{0.8,1.0,0.8}
\definecolor{red}{rgb}{0.8,0,0}
\definecolor{green}{rgb}{0,0.8,0}
\definecolor{blue}{rgb}{0,0,1}
\newcommand{\mytodo}[1]{\todo[inline,color=lightgreen]{TODO:#1}}
\newcommand{\myremark}[1]{\todo[inline, color=lightgreen]{\textbf{Remark:} #1}}

\title{The VNN-LIB standard for benchmarks\\2025}

\author[1]{Stefano Demarchi}
\author[2]{Dario Guidotti}
\author[2]{Luca Pulina}
\author[1]{Armando Tacchella}

\author[3]{Ann Roy}
\author[3]{Allen Antony}
\author[3]{Matthew Daggitt}

\affil[1]{University of Genoa, Viale Causa 13, 16145 Genoa, Italy}
\affil[2]{University of Sassari, Via Roma 151, 07100 Sassari, Italy}
\affil[3]{University of Western Australia, 35 Stirling Hwy, Crawley WA 6009, Australia}
  
\begin{document}

\maketitle

\begin{abstract}
This document presents VNN-LIB, a standard that formalises the query language for neural network verifiers. The standard uses the 
Open Neural Network Exchange (ONNX) format for model description and builds upon the Satisfiability Modulo Theories Library (SMT-LIB) 
format for query specification. Key among the standard is a formally defined syntax and semantics, complimentary tooling, as well as a 
command-line interface for verifiers. The goal is to foster greater robustness and interoperability in the neural network verification 
landscape.
\end{abstract}


\section*{Introduction}

As neural networks become increasingly integral to safety-critical and high societal impact applications\cite{1, 2, 3}, 
the need for robust verification of their properties is paramount. While neural networks have shown exceptional performance, 
they are susceptible to issues like adversarial examples\cite{3} and discriminatory behaviour\cite{4}.

Since 2023, the VNN-LIB standard\cite{5} has served as a de facto specification language for queries in the neural network verification 
community, notably supporting the annual VNN-COMP\cite{7}. It was conceived to facilitate the standardisation of solver interfaces and 
enable the collection of benchmarks in a common format. This document introduces a new version of the VNN-LIB standard which addresses 
shortcomings in expressiveness, conciseness, and formal rigour\cite{5, 7}.

This document is structured as follows. Chapter~\ref{sec:model} revisits the guidelines for model specification using ONNX.\@
Chapter~\ref{sec:specification_language} details the VNN-LIB query specification language, with Section~\ref{sec:smtlib} describing the 
SMT-LIB first-order logic specification language which VNN-LIB is inspired from, Section~\ref{sec:syntax} describing the formal syntax of VNN-LIB and 
Section~\ref{sec:semantics} discussing its semantics. Chapter~\ref{sec:query_categories} outlines different characteristics of queries, namely linearity, 
reachability, and normality. Chapter~\ref{sec:solver_interface} introduces a standard command line interface for verifiers. Finally, Chapter~\ref{sec:examples} 
presents examples illustrating how to use VNN-LIB to specify constraints.

\chapter{Network Representation}\label{sec:model}
\input{onnx_modelzoo_report}

\chapter{Query Specification Language}\label{sec:specification_language}

\section{SMT-LIB Property Language}\label{sec:smtlib}
\input{smtlib_property}

\section{Syntax}\label{sec:syntax}
% Define some colors for listings
\definecolor{codegreen}{rgb}{0,0.6,0}
\definecolor{codegray}{rgb}{0.5,0.5,0.5}
\definecolor{codepurple}{rgb}{0.58,0,0.82}
\definecolor{backcolour}{rgb}{0.95,0.95,0.92}
\definecolor{keywordblue}{rgb}{0.13,0.13,1}
\definecolor{stringred}{rgb}{0.8,0,0}

% Listings style for Bash/CLI commands
\lstdefinestyle{lbnf}{
    backgroundcolor=\color{backcolour},   
    commentstyle=\color{codegreen},
    keywordstyle=\color{keywordblue}\bfseries,
    numberstyle=\tiny\color{codegray},
    stringstyle=\color{stringred},
    basicstyle=\ttfamily\footnotesize,
    breakatwhitespace=false,         
    breaklines=true,                 
    captionpos=b,                    
    keepspaces=true,                 
    numbers=left,                    
    numbersep=5pt,                  
    showspaces=false,                
    showstringspaces=false,
    showtabs=false,                  
    tabsize=1,
    frame=lines,
    xleftmargin=2em,
    framexleftmargin=1.5em,
    escapeinside={\%*}{*} 
}


The syntax of VNN-LIB is formally defined using Labelled Backus-Naur Form (LBNF)\cite{8}. LBNF is a variant of BNF that allows for 
annotations (labels) on productions, facilitating the automatic generation of abstract syntax trees, parsers, and other language processing tools. 
This formal grammar provides a rigorous foundation for the language, eliminating ambiguities present in previous versions and ensuring consistent 
parsing across different tools.

The full LBNF grammar for VNN-LIB is provided in the Appendix. The following subsections highlight key syntactic constructs of the language,
with examples illustrating their usage.

\subsection{Network Declaration}
VNN-LIB supports the definition of one or more neural networks within a single specification file. This is crucial for properties that need to refer to
multiple networks. A network declaration is introduced by the keyword \texttt{declare-network}, followed by a user-defined variable name for the network, 
and then its associated input, hidden (hidden), and output variable declarations. All variables are declared inside of network declarations and variable 
names must be unique within the scope of the entire VNN-LIB specification.

\begin{lstlisting}[
    caption=Network Definition Example, 
    style=lbnf,
    label={lst:network-definition}
]
(declare-network acc
    (declare-input X Real 3)
    (declare-hidden N Real 1 2 onnx-node:"node_name_in_onnx")
    (declare-output Y Real 3)
) 
\end{lstlisting}

For instance, Listing~\ref{lst:network-definition} declares a network named \texttt{acc}, with an input tensor \texttt{X}, hidden tensor \texttt{N}, 
and output tensor \texttt{Y}. 

\subsection{Input and Output Variable Declarations}

An input variable is declared using the \texttt{declare-input} keyword, followed by a variable name, its element type (e.g., \texttt{Real}, \texttt{int8}), 
and a space-seperated list of integers representing the shape of the tensor. Similarly, an output variable uses the \texttt{declare-output} keyword. Multiple 
input and output variables can be declared within a single network declaration, and the order of declaration determines the mapping to the nodes in the ONNX model.
The variables may be explicitly associated with ONNX node names using the \texttt{onnx-node} keyword and a string identifer, only under the condition that all input 
and output variables are associated with an ONNX node name.

\begin{lstlisting}[
  caption=Input and Output Definition Example,
  style=lbnf,
  label={lst:input-output}
]
(declare-input X Real 1 28 28) 
\end{lstlisting}

For example, Listing~\ref{lst:input-output} declares an input tensor named \texttt{X} of real numbers with shape $1 \times 28 \times 28$. 

\subsection{Hidden Node Declarations}

A hidden node is declared using the \texttt{declare-hidden} keyword. This declaration includes a variable name for use within the VNN-LIB specification, 
its element type, its tensor shape, and crucially, a string identifier that specifies the corresponding node name in the ONNX graph. The ability to declare hidden nodes
allows for properties to reference key intermediate computations within the network, such as encoding features, attention mechanisms, or other internal states. Multiple
hidden nodes can be trivially declared within a single network declaration.

\begin{lstlisting}[
    caption=hidden Node Declaration Example, 
    style=lbnf,
    label={lst:hidden-node}
]
(declare-hidden H1 Real 100 onnx-node:"layer3/relu_out") 
\end{lstlisting}

Listing~\ref{lst:hidden-node} declares an hidden variable \texttt{H1} of type \texttt{Real} with a shape of `(100)`. The \texttt{onnx-node} attribute 
specifies that this variable corresponds to the tensor named \texttt{"layer3/relu\_out"} in the associated ONNX graph. 

\subsection{Assertion Specification}
VNN-LIB supports quantifier-free logic formulas. Asertions are defined using parenthesised \texttt{(assert\ldots)} expressions, and follows an SMT-LIB-like syntax with the 
operand preceding its arguments. The operands can be logical operators (e.g., \texttt{and}, \texttt{or}) or arithmetic operators (e.g., \texttt{+}, \texttt{-}, \texttt{*}).
An assertion is a logical formula that consists of logical and arithmetic operations over one or more elements of the declared tensors.

\paragraph{Matrix Notation}
Let $X \in I$ be an $n$-dimensional tensor in some generic input domain $I = I^{d_1 \times \cdots \times d_n}$. The ``matrix notation'' represents a specific 
element $x_{i_1, i_2, \dots, i_n}$ of the tensor $X$ as \texttt{X\_$i_1$-$i_2$-\dots-$i_n$}, where $i_1, \dots, i_n$ are the indices of the element of interest in the 
dimensions $d_1, \dots, d_n$. To better clarify, if we consider the 1-D tensor $X \in I^n$, the 2-D tensor $Y \in I^{n \times m}$, and the 3-D tensor 
$Z \in I^{n \times m \times p}$, we will have the following representations:
\begin{itemize}
    \item \texttt{X\_0}, \texttt{X\_1}, \dots, \texttt{X\_$i$}, \dots, \texttt{X\_$n$};
    \item \texttt{Y\_0-0}, \texttt{Y\_0-1}, \dots, \texttt{Y\_$i$-$j$}, \dots, \texttt{Y\_$n$-$m$};
    \item \texttt{Z\_0-0-0}, \texttt{Z\_0-0-1}, \dots, \texttt{Z\_$i$-$j$-$k$}, \dots, \texttt{Z\_$n$-$m$-$p$};
\end{itemize}
In such a representation, \texttt{Z\_$i$-$j$-$k$} corresponds to the element $z_{i,j,k}$ of the tensor $Z$. 

\paragraph{Assertion Example}
For example, Listing~\ref{lst:assertion-example} asserts that for a given range of the input neuron $A_1$, the output neuron $B_0$ 
is greater than another output neuron $B_1$. More complex properties, including those referencing multiple networks or hidden nodes, 
can be constructed using these foundational elements.

\begin{lstlisting} [
	caption=Assertion Example, 
	style=lbnf,
    label={lst:assertion-example}
]
(assert 
    (and 
        (and (>= A_0 0.0) (<= B_0 1.0)) 
        (> B_0 B_1)
    )
)
\end{lstlisting}


\section{Semantics}\label{sec:semantics}

% ---------------------------------------------------------------

\chapter{Query Categories}\label{sec:query_categories}

\section{Linearity}\label{sec:linearity}

\section{Reachability}\label{sec:reachability}

\section{Disjunctive Normal Form}\label{sec:dnf}

% ---------------------------------------------------------------

\chapter{Solver Interface}\label{sec:solver_interface}
\section{Functionalities}\label{sec:functionalities}
% Define some colors for listings
\definecolor{codegreen}{rgb}{0,0.6,0}
\definecolor{codegray}{rgb}{0.5,0.5,0.5}
\definecolor{codepurple}{rgb}{0.58,0,0.82}
\definecolor{backcolour}{rgb}{0.95,0.95,0.92}
\definecolor{keywordblue}{rgb}{0.13,0.13,1}
\definecolor{stringred}{rgb}{0.8,0,0}

% Listings style for Bash/CLI commands
\lstdefinestyle{bash}{
    backgroundcolor=\color{backcolour},   
    commentstyle=\color{codegreen},
    keywordstyle=\color{keywordblue}\bfseries,
    numberstyle=\tiny\color{codegray},
    stringstyle=\color{stringred},
    basicstyle=\ttfamily\footnotesize,
    breakatwhitespace=false,         
    breaklines=true,                 
    captionpos=b,                    
    keepspaces=true,                 
    numbers=left,                    
    numbersep=5pt,                  
    showspaces=false,                
    showstringspaces=false,
    showtabs=false,                  
    tabsize=2,
    frame=lines,
    xleftmargin=2em,
    framexleftmargin=1.5em,
    escapeinside={\%*}{*} 
}

\section{Introduction}
This chapter specifies the command-line interface (CLI) for neural network verifiers compliant with the VNNLIB-2.0 standard. The goal is to provide a consistent and 
predictable interface for users and automated tools to interact with different verifiers. The CLI supports querying verifier capabilities, listing supported operations, 
and invoking the verification process with various configurations.

All verifiers adhering to this specification should expose an executable, referred to in this document as \texttt{vnn\_verifier}.

\section{General Usage}
The general syntax for interacting with the verifier via the CLI is:
\begin{lstlisting}[style=bash, numbers=none, frame=none, backgroundcolor=\color{white}]
vnn_verifier [global-options] <command> [command-options] 
	[arguments] \end{lstlisting}

\section{Global Options}
These options are applicable to the main \texttt{vnn\_verifier} executable and potentially to all commands.
\begin{itemize}
    \item \texttt{--help}, \texttt{-h}: Display help information for the CLI or for a specific command if used like \texttt{vnn\_verifier <command> --help}.
    \item \texttt{--name}, \texttt{-v}: Display the verifier's name.
    \item \texttt{--version}, \texttt{-V}: Display the verifier's version.
\end{itemize}

\section{Commands}

\subsection{\texttt{verify}}
\begin{itemize}
    \item \textbf{Description}: Invokes the verification process on a given neural network model and a VNNLIB property specification.
    \item \textbf{Usage}:
    \begin{lstlisting}[style=bash]
vnn_verifier verify
	 --network <name1>:<net1_path> 
	[--network <name2>:<net2_path> ...] 
	 --property <vnnlib_filepath> [options] \end{lstlisting}
    \item \textbf{Options}:
    \begin{longtable}{@{}>{\raggedright\arraybackslash\ttfamily}p{0.28\textwidth}% Option column
                    >{\raggedright\arraybackslash\ttfamily}p{0.12\textwidth}% Shorthand column
                    >{\raggedright\arraybackslash\ttfamily}p{0.20\textwidth}% Argument column
                    >{\raggedright\arraybackslash}p{0.32\textwidth}@{}}% Description column (normal font for readability)
        \toprule
        Option & Shorthand & Argument & Description \\
        \midrule
        \bottomrule
        \texttt{--network} & \texttt{-n} & \texttt{<name>:<path>} & \textbf{(Required)} Identifier for the network and path to the ONNX model file. \\
        \texttt{--property} & \texttt{-p} & \texttt{<path>} & \textbf{(Required)} Path to the VNNLIB 2.0 property file. \\
        \texttt{--timeout} & \texttt{-t} & \texttt{<seconds>} & Maximum verification time in integer seconds. \\
        \texttt{--dataset} & \texttt{-d} & \texttt{<path>} & Path to a dataset file (optional, for models that require input data).\\
        \texttt{--output} & \texttt{-o} & \texttt{<path>} & Path to save the satisfying assignment (if applicable). \\
    \end{longtable}
    \item \textbf{Standard Output/Error}:
    \begin{itemize}
        \item The primary result is printed to \texttt{stdout}: \texttt{SAT}, \texttt{UNSAT}, \texttt{TIMEOUT}, \texttt{UNKNOWN}.
        \item If \texttt{SAT}, the generated satisfying assignment will be printed to \texttt{stdout} or saved to the specified file.
        \item Detailed logs or error messages are sent to \texttt{stderr}.
    \end{itemize}
\end{itemize}

\myremark{The \texttt{<name>} in the \texttt{--network} option is a user-defined identifier for the network, which is used in the VNNLIB property file to refer to this specific network.}

\section{Granular Capability Options}
This section details additional global options for querying individual verifier capabilities. These options provide a convenient way to check for specific features of the verifier.

\subsection{\texttt{supports-multiple-networks}}
\begin{itemize}
    \item \textbf{Shorthand}: \texttt{mn}
    \item \textbf{Description}: Checks if the verifier supports properties defined over multiple neural networks. 
    \item \textbf{Usage}: \texttt{vnn\_verifier --supports-multiple-networks}
    \item \textbf{Output}: Prints "true" or "false" to stdout.
\end{itemize}

\subsection{\texttt{supports-multiple-input-output}}
\begin{itemize}
    \item \textbf{Shorthand}: \texttt{mio}
    \item \textbf{Description}: Checks if the verifier supports models with multiple input and/or output tensors. 
    \item \textbf{Usage}: \texttt{vnn\_verifier --supports-multiple-io}
    \item \textbf{Output}: Prints "true" or "false" to stdout.
\end{itemize}

\subsection{\texttt{supports-hidden-nodes}}
\begin{itemize}
    \item \textbf{Shorthand}: \texttt{hn}
    \item \textbf{Description}: Checks if the verifier supports referencing hidden layer nodes in VNNLIB properties. 
    \item \textbf{Usage}: \texttt{vnn\_verifier --supports-hidden-nodes}
    \item \textbf{Output}: Prints "true" or "false" to stdout.
\end{itemize}

\subsection{\texttt{requires-dnf}}
\begin{itemize}
    \item \textbf{Shorthand}: \texttt{dnf}
    \item \textbf{Description}: Checks if the verifier requires assertions to be in Disjunctive Normal Form (DNF). 
    \item \textbf{Usage}: \texttt{vnn\_verifier --requires-dnf}
    \item \textbf{Output}: Prints "true" or "false" to stdout.
\end{itemize}

\subsection{\texttt{supports-strict-comparison}}
\begin{itemize}
    \item \textbf{Shorthand}: \texttt{sc}
    \item \textbf{Description}: Checks if the verifier supports strict inequalities (\texttt{<}, \texttt{>}). 
    \item \textbf{Usage}: \texttt{vnn\_verifier --supports-strict-comparison}
    \item \textbf{Output}: Prints "true" or "false" to stdout.
\end{itemize}

\subsection{\texttt{requires-linear-complexity}}
\begin{itemize}
    \item \textbf{Shorthand}: \texttt{lc}
    \item \textbf{Description}: Checks if the verifier's supported assertions are limited to linear arithmetic. 
    \item \textbf{Usage}: \texttt{vnn\_verifier --requires-linear-complexity}
    \item \textbf{Output}: Prints "true" or "false" to stdout.
\end{itemize}

\subsection{\texttt{is-reachability-based}}
\begin{itemize}
    \item \textbf{Shorthand}: \texttt{rb}
    \item \textbf{Description}: Checks if the verifier is a reachability-based tool. 
    \item \textbf{Usage}: \texttt{vnn\_verifier --is-reachability-based}
    \item \textbf{Output}: Prints "true" or "false" to stdout.
\end{itemize}

\subsection{\texttt{supported-domains}}
\begin{itemize}
    \item \textbf{Shorthand}: \texttt{sd}
    \item \textbf{Description}: Prints a newline-separated list of the abstract input domain representations supported by the verifier. 
    \item \textbf{Usage}: \texttt{vnn\_verifier --list-input-domains}
    \item \textbf{Example Output}:
    \begin{lstlisting}[style=bash, numbers=none, frame=none, backgroundcolor=\color{white}]
Box
Zonotope
Polytope
...
    \end{lstlisting}
\end{itemize}

\subsection{\texttt{supported-types}}
\begin{itemize}
    \item \textbf{Shorthand}: \texttt{st}
    \item \textbf{Description}: Prints a newline-separated list of supported ONNX data types.
    \item \textbf{Usage}: \texttt{vnn\_verifier --list-supported-types}
    \item \textbf{Example Output}:
    \begin{lstlisting}[style=bash, numbers=none, frame=none, backgroundcolor=\color{white}]
float16
float32
...
    \end{lstlisting}
\end{itemize}

\subsection{\texttt{supported-onnx-ops}}
\begin{itemize}
    \item \textbf{Shorthand}: \texttt{so}
    \item \textbf{Description}: Prints a newline-separated list of ONNX operator types (e.g., ``Conv'', ``Relu'', ``Gemm'') that are supported by the verifier. 
	This helps users determine if their ONNX model is compatible with the verifier.
    \item \textbf{Usage}: \texttt{vnn\_verifier --list-onnx-ops}
    \item \textbf{Example Output}:
    \begin{lstlisting}[style=bash, numbers=none, frame=none, backgroundcolor=\color{white}]
Conv
Relu
MatMul
Gemm
Add
Flatten
...
    \end{lstlisting}
\end{itemize}

\section{Satisfying Assignments}\label{sec:satisfying_assignments}

% ---------------------------------------------------------------

\chapter{Examples}\label{sec:examples}

\appendix
\chapter{VNN-LIB LBNF Grammar}\label{app:lbnf_grammar}
\myremark{The full LBNF grammar, as developed using BNFC, should be included here.}

\begin{thebibliography}{9} 

\bibitem{1} 
D. Tang, B. Qin, and T. Liu, ``Document modelling with gated recurrent neural network for sentiment classification,'' in \emph{Proceedings of the 2015 Conference on Empirical Methods in Natural Language Processing}, 2015, pp. 1422--1432.

\bibitem{2} 
M. Bojarski, et al., ``End to end learning for self-driving cars,'' \emph{arXiv preprint arXiv:1604.07316}, 2016.

\bibitem{3} 
C. Szegedy, et al., ``Intriguing properties of neural networks,'' \emph{arXiv preprint arXiv:1312.6199}, 2013.

\bibitem{4} 
P. Zhang et al., ``White-box fairness testing through adversarial sampling,'' in \emph{2020 IEEE/ACM 42nd International Conference on Software Engineering (ICSE)}, New York, NY, USA:\@ ACM, 2020, pp. 949--960.\'doi: \href{https://doi.org/10.1145/3377811.3380331}{10.1145/3377811.3380331}.

\bibitem{5} 
S. Demarchi, D. Guidotti, L. Pulina, and A. Tacchella, ``Supporting Standardization of Neural Networks Verification with VNN-LIB and CoCoNet,'' in \emph{Proc. 6th Int. Workshop on Formal Methods for ML-Enabled Autonomous Systems (FoMLAS 2023)}, 2023, pp. 47--58.

\bibitem{6} 
C. Brix, S. Bak, C. Liu, and T. T. Johnson, ``The Fourth International Verification of Neural Networks Competition (VNN-COMP 2023): Summary and Results,'' 2023, doi: \href{https://doi.org/10.48550/arxiv.2312.16760}{10.48550/arxiv.2312.16760}.

\bibitem{7} 
L. C. Cordeiro et al., ``Neural Network Verification is a Programming Language Challenge,'' 2025, doi: \href{https://doi.org/10.48550/arxiv.2501.05867}{10.48550/arxiv.2501.05867}.

\bibitem{8} 
M. Forsberg and A. Ranta, ``The Labelled BNF Grammar Formalism,'' Department of Computing Science, Chalmers University of Technology and the University of Gothenburg, Gothenburg, Sweden, Feb. 11, 2005. [Online]. Available: \url{https://bnfc.digitalgrammars.com/LBNF-report.pdf}

\end{thebibliography}

\end{document}
