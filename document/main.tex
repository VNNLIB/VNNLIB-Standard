\documentclass[12pt,a4paper]{report}

\usepackage[utf8]{inputenc}
\usepackage[english]{babel}
\usepackage{amsmath}
\usepackage{amsfonts}
\usepackage{amssymb}
\usepackage{graphicx}
\usepackage{hyperref}
\usepackage{todonotes}
\usepackage{xcolor}
\usepackage{listings}
\usepackage{authblk}
\usepackage{algorithm, algpseudocode}
\usepackage{geometry}
\usepackage{booktabs} 
\usepackage{longtable} 
\usepackage{array}


\newcommand{\ie}{\textit{i.e.}}
%customized todo
\definecolor{lightgreen}{rgb}{0.8,1.0,0.8}
\definecolor{red}{rgb}{0.8,0,0}
\definecolor{green}{rgb}{0,0.8,0}
\definecolor{blue}{rgb}{0,0,1}
\newcommand{\mytodo}[1]{\todo[inline,color=lightgreen]{TODO:#1}}
\newcommand{\myremark}[1]{\todo[inline, color=lightgreen]{\textbf{Remark:} #1}}

\title{The VNN-LIB standard for benchmarks\\2025}

\author[1]{Stefano Demarchi}
\author[2]{Dario Guidotti}
\author[2]{Luca Pulina}
\author[1]{Armando Tacchella}

\author[3]{Ann Roy}
\author[3]{Allen Antony}
\author[3]{Matthew Daggitt}

\affil[1]{University of Genoa, Viale Causa 13, 16145 Genoa, Italy}
\affil[2]{University of Sassari, Via Roma 151, 07100 Sassari, Italy}
\affil[3]{University of Western Australia, 35 Stirling Hwy, Crawley WA 6009, Australia}
  
\begin{document}

\maketitle

\begin{abstract}
This document presents VNN-LIB, a standard that formalises the query language for neural network verifiers. The standard uses the 
Open Neural Network Exchange (ONNX) format for model description and builds upon the Satisfiability Modulo Theories Library (SMT-LIB) 
format for query specification. Key among the standard is a formally defined syntax and semantics, complimentary tooling, as well as a 
command-line interface for verifiers. The goal is to foster greater robustness and interoperability in the neural network verification 
landscape.
\end{abstract}


\section*{Introduction}

As neural networks become increasingly integral to safety-critical and high societal impact applications\cite{1, 2, 3}, 
the need for robust verification of their properties is paramount. While neural networks have shown exceptional performance, 
they are susceptible to issues like adversarial examples\cite{3} and discriminatory behaviour\cite{4}.

Since 2023, the VNN-LIB standard\cite{5} has served as a de facto specification language for queries in the neural network verification 
community, notably supporting the annual VNN-COMP\cite{7}. It was conceived to facilitate the standardisation of solver interfaces and 
enable the collection of benchmarks in a common format. This document introduces a new version of the VNN-LIB standard which addresses 
shortcomings in expressiveness, conciseness, and formal rigour\cite{5, 7}.

This document is structured as follows. Chapter~\ref{sec:model} revisits the guidelines for model specification using ONNX.\@
Chapter~\ref{sec:specification_language} details the VNN-LIB query specification language, with Section~\ref{sec:smtlib} describing the 
SMT-LIB first-order logic specification language which VNN-LIB is inspired from, Section~\ref{sec:syntax} describing the formal syntax of VNN-LIB and 
Section~\ref{sec:semantics} discussing its semantics. Chapter~\ref{sec:query_categories} outlines different characteristics of queries, namely linearity, 
reachability, and normality. Chapter~\ref{sec:solver_interface} introduces a standard command line interface for verifiers. Finally, Chapter~\ref{sec:examples} 
presents examples illustrating how to use VNN-LIB to specify constraints.

\chapter{Network Representation}\label{sec:model}
Here we describe the operators that are officially supported, i.e.,
those operators supplied by the ONNX model 
zoo~\footnote{https://github.com/onnx/models} that allow to represent
the majority of the models in the zoo limiting as much as possible the
variety. We consider the model zoo as representative enough for the 
kind of model architectures, and operators, that are commonly used in
the Machine Learning community.

The following operators cover almost every benchmark provided in the
VNN-COMP repositories for sequential networks; other kinds of networks
(ResNet, Recurrent, etc.) are often based on "exotic" and, in general,
peculiar operators that do not lie in this list.

\myremark{While the standard supports the operators, it is strongly 
	unadvised to include pre-processing in the benchmark model, e.g., 
	normalization and flattening, since the properties should match
	the first node with the same input dimension.}

\begin{itemize}
	\item \emph{Add (Add)} operator performs the element-wise sum of
		a tensor and a scalar. We strongly encourage to use the 
		\textit{Gemm} operator when paired with \textit{MatMul}.
	
	\item \emph{AveragePool (Average Pooling)} operator
	  supports downsampling with averaging.
	
	\item \emph{BatchNormalization (Batch
	  Normalization)} operator supports adjusting and scaling the
	  activations functions, and it is expressive enough to represent
	  general batch normalization.
	  
	\item \emph{Concat (Concatenation)} operator concatenates a list
		of tensors into a single tensor, with the same shape except for
		the axis to concatenate on.	
	
	\item \emph{Conv (Convolutional)} operator supports
	  all the attributes to encode a generic convolutional layer.
	
	\item \emph{Dropout (Dropout)} operator supports
	  random dropping of units (during training). This operator should not
	  appear on trained models.  
	  
	\item \emph{Flatten (Flatten)} this operator converts multidimensional
	  arrays (tensors) to single dimensional ones; it is used instead of
	  \emph{Reshape} in some of the models in the zoo.
	  
	\item \emph{Gemm (General Matrix Multiplication)}
	  operator encodes matrix multiplication possibly with a scalar
	  coefficient and the addition of another matrix; as such \emph{Gemm}
	  can encode fully connected layers in neural networks.
	
	\item \emph{LRN (Local Response Normalization)}
	  operator supports normalization over local input regions; it is uatilized
	  in Alexnet and derived networks.
	  
	\item \emph{MatMul (Matrix Multiplication)} operator performs a
		numPy-like matrix multiplication. We strongly encourage to use
		the \textit{Gemm} operator.
	  
	\item \emph{MaxPool (Maximum pooling)} operator supports
	  downsampling with maximization.
	
	\item \emph{ReLU (Rectified Linear Unit)} operator
	  encodes the corresponding activation function $\sigma(x) = \max(0, x)$.
	  
	\item \emph{Reshape (Reshape)} operator supports
	  reshaping of the tensor's dimensions.
	  
	\item \emph{Sigmoid (Logistic Unit)} operator
	  encodes the corresponding activation function $\sigma(x) =
	  \frac{1}{1 + e^{-x}}$.
	  
	\item \emph{SoftMax (Softmax Unit)} operator transforms
	  vectors into probabilities, e.g., for selecting among different
	  classes and it is commonly utilized in state of the art
	  networks.
	  
	\item \emph{Sub (Sub)} operator performs element-wise binary subtraction
		between two tensors.
	
	\item \emph{Unsqueeze (Unsqueeze)} operator removes dimensions of size
	  1 from tensors, and it is utilized, e.g., in \emph{Densenet} and
	  \emph{Inception2}.
\end{itemize}

%% At a high level, networks can be seen as functions 
%% $\nu : I^n \to O^m$, mapping an $n$-dimensional \emph{input domain}
%% $I^n$ ($n > 0$) to a $m$-dimensional \emph{output domain} $O^m$ ($m
%% >0$). We argue that this representation captures most cases of practical
%% interest.
%% For instance, a network computing an approximation
%% of some function $f: \mathbb{R}^n \to \mathbb{R}$ would have $I = O =
%% \mathbb{R}$, whereas a network classifying 8-bit images of size $h \times v$ in
%%  two classes would be defined as ${\nu: \{0,\ldots,255\}^{h \cdot v}
%%    \to \{0, 1\}}$ with $I=\{0, \ldots, 255\}$ and $O = \{0,1\}$.




\chapter{Query Specification Language}\label{sec:specification_language}

\section{SMT-LIB Property Language}\label{sec:smtlib}
Inputs and outputs of operators are \emph{tensors}, i.e.,
multidimensional arrays over some domain, usually numerical. 
If we let $\mathbb{D}$ be any such domain, a $k$-dimensional 
tensor on $\mathbb{D}$ is denoted as $x \in \mathbb{D}^{n_1 
	\times \ldots \times n_k}$.
For example, a vector of $n$ real numbers is a 1-dimensional
tensor $x \in \mathbb{R}^n$, whereas a matrix of $n \times n$ 
Booleans is a 2-dimensional tensor $x \in \mathbb{B}^{n 
	\times n}$ with $\mathbb{B} = \{0, 1\}$. A specific element 
of a tensor can be singled-out via \emph{subscripting}. 

Given a $k$-dimensional tensor $x \in \mathbb{D}^{n_1 \times 
	\ldots \times n_k}$, the element $x_{i_1, \ldots, i_k} \in 
	\mathbb{D}$ is a scalar corresponding to the indexes 
${i_1, \ldots, i_k}$. For example, in a vector of real numbers 
$x \in \mathbb{R}^n$, $x_1$ is the first element, $x_2$ the second 
and so on. In a matrix of Boleans $x \in \mathbb{B}^{n \times
  n}$, $x_{1,1}$ is the first element of the first row, $x_{2,1}$ 
is the first element of the second and so on.

An \emph{operator} $f$ is a function on tensors 
$f: \mathbb{D}^{n_{1} \times n_h} \to \mathbb{D}^{m_{1} \times m_k}$
where $h$ is the dimension of the input tensor and $k$ is the 
dimension of the output tensor. Given a set $F = \{f_1, \ldots, 
	f_p\}$ of $p$ operators, a \emph{feedforward neural network}
is a function $\nu = f_p(f_{p-1}(\ldots f_2(f_1(x))\ldots))$ obtained
through the composition of the operators in $F$ assuming that the 
dimensions of their inputs and outputs are \emph{compatible}, i.e.,
if the  output of $f_i$ is a $k$-dimensional tensor, then the input
of $f_{i+1}$ is also a $k$-dimensional tensor, for all $1 \leq i < p$.

Given a neural network $\nu : \mathbb{D}^{n_{1} \times n_h} \to
\mathbb{D}^{m_{1} \times m_k}$ built on the set of operators $\{f_1,
\ldots, f_p\}$, let $x \in \mathbb{D}^{n_{1} \times n_h}$ denote
the input of $\nu$ and $y_1, \ldots, y_p$ denote the outputs of the
operators $f_1, \ldots, f_p$ --- therefore $y_p$ is also the output
$y$ of $\nu$. We assume that, in general, a \emph{property} is a first
order formula $P(x, y_1, \ldots y_p)$ which should be satisfied given 
$\nu$. More formally, given $p$ bounded sets $X_1, \ldots, X_p$ in $I$ 
such that $\Pi = \bigcup_{i=1}^p X_i$ and $s$ bounded sets $Y_1, 
\ldots, Y_s$ in $O$ such that $\Sigma = \bigcup_{i=1}^s Y_i$, we wish
to prove that  
\begin{equation}
	\label{eq:verif}
	\forall x \in \Pi \rightarrow \nu(x) \in \Sigma.
\end{equation}
The definition of the property given in equation (\ref{eq:verif})
consists of a \textit{pre-}condition $x \in \Pi$ and a 
\textit{post-}condition $\nu(x) \in \Sigma$. The 
\textit{pre-}condition encodes the bounds of the input space, i.e.,
bounds the variables that are fed to the network, and the 
\textit{post-}condition defines the safe zone, outside which the 
verification task fails.

The SMT-LIB language is a well-known language used to formalize 
Satisfiability Modulo Theories problems, and is expressive enough to
represent the verification properties of interest. In this language, 
it is possible to define both the \textit{pre-} and 
\textit{post-}conditions at once, by defining the variables for the
input and the output of the neural network. In the following we
show some examples of networks and corresponding properties in the
SMT-LIB language.

\myremark{Note that the input and output variable names should match
	the identifiers of the input and of the last node in the network.}

\section{Syntax}\label{sec:syntax}
The syntax of VNN-LIB 2.0 is formally defined using Labelled Backus-Naur Form (LBNF)\cite{8}. LBNF is a variant of BNF that allows for 
annotations (labels) on productions, facilitating the automatic generation of abstract syntax trees, parsers, and other language processing tools. 
This formal grammar provides a rigorous foundation for the language, eliminating ambiguities present in previous versions and ensuring consistent 
parsing across different tools.

The full LBNF grammar for VNN-LIB 2.0 is provided in the Appendix. The following subsections highlight key syntactic constructs introduced 
or modified in VNN-LIB 2.0, simplified for readability.

\subsection{Network Definitions}
VNN-LIB 2.0 supports the definition of one or more neural networks within a single specification file. This is crucial for properties involving 
multiple networks, such as checking equivalence or performing compositional verification.

A network definition is introduced by the keyword \texttt{declare-network}, followed by a user-defined variable name for the network, and then its 
associated input, intermediate (hidden), and output variable declarations, all enclosed in parentheses.

The LBNF rule can be generally represented as:
\begin{lstlisting}[
    caption=Network Definition Rule, 
    basicstyle=\ttfamily\footnotesize,
    breaklines=true,               
    breakatwhitespace=false,       
    breakindent=2em,                
    postbreak=\mbox{\textcolor{red}{$\hookrightarrow$}\space} 
]
NetworkDefinition ::= "(declare-network" VariableName InputDefinition+ IntermediateDefinition* OutputDefinition+ ")" ;
\end{lstlisting}
Here, \texttt{VariableName} is an identifier for the network.\texttt{InputDefinition+} indicates one or more input definitions, \texttt{IntermediateDefinition*} 
indicates zero or more intermediate (hidden) node definitions, and \texttt{OutputDefinition+} indicates one or more output definitions. These components are detailed below.

\subsection{Input and Output Variable Declarations}
In VNN-LIB 1.0, input and output variables were implicitly defined by the first and last nodes of the ONNX model, respectively. However, this approach lacked flexibility and expressiveness,
especially for networks with multiple inputs or outputs, or when properties needed to refer to specific nodes within the ONNX network. To support networks with multiple inputs or outputs, 
and to provide more expressive tensor declarations, VNN-LIB 2.0 introduces explicit declarations for input and output variables.

An input variable is declared using the \texttt{declare-input} keyword, followed by a variable name, its element type (e.g., \texttt{Real}, \texttt{int8}), and a space-seperated list of integers 
representing the shape of the tensor. Similarly, an output variable uses the \texttt{declare-output} keyword.

The LBNF rules are:
\begin{lstlisting}[
    caption=Network Definition Rule, 
    basicstyle=\ttfamily\footnotesize,
    breaklines=true,               
    breakatwhitespace=false,       
    breakindent=2em,                
    postbreak=\mbox{\textcolor{red}{$\hookrightarrow$}\space} 
]
InputDefinition ::= "(declare-input" VariableName ElementType Int* ")" ;
OutputDefinition ::= "(declare-output" VariableName ElementType Int* ")" ;
\end{lstlisting}
For example, \texttt{(declare-input X Real 1 28 28)} declares an input tensor named \texttt{X} of real numbers with shape $1 \times 28 \times 28$. This syntax allows for tensor-level 
declarations directly, improving conciseness compared to VNN-LIB 1.0, where each tensor element was required to be declared individually.

\myremark{By default, the declared variables correspond to the nodes in the associated ONNX model by their order of declaration. You may specify the ONNX node names explicitly
using the \texttt{onnx-node} keyword under the condition that an ONNX name is provided for each variable.}

\subsection{Intermediate (Hidden) Node Declarations}
A significant extension in VNN-LIB 2.0 is the ability to reference internal (hidden) nodes of a neural network. This is essential for specifying properties on intermediate 
layers, verifying composite architectures (e.g., encoder-decoder models), or for observer-controller systems.

An intermediate node is declared using the \texttt{declare-intermediate} keyword. This declaration includes a variable name for use within the VNN-LIB specification, 
its element type, its tensor shape, and crucially, a string identifier that specifies the corresponding node name in the ONNX graph.

The LBNF rule is:
\begin{lstlisting}[
    caption=Network Definition Rule, 
    basicstyle=\ttfamily\footnotesize,
    breaklines=true,               
    breakatwhitespace=false,       
    breakindent=2em,                
    postbreak=\mbox{\textcolor{red}{$\hookrightarrow$}\space} 
]
IntermediateDefinition ::= "(declare-intermediate" VariableName ElementType Int* "onnx-node" ":" String ")" ;
\end{lstlisting}
For example, \texttt{(declare-intermediate H1 Real 100 onnx-node:``layer3/relu\_out'')} declares an intermediate variable \texttt{H1} corresponding to the 
ONNX tensor named ``layer3/relu\_out''.

\subsection{Assertion Specification}
VNN-LIB 2.0 continues to support first-order logic formulae as in VNNLIB-1.0. Asertions follow an SMT-LIB-like syntax, defined using parenthesised 
\texttt{(assert\ldots)} expressions. An assertion consists of logical and arithmetic operations over one or more elements of the declared tensors.

\paragraph{Matrix Notation}
Let $X \in I$ be an $n$-dimensional tensor in some generic input domain $I = I^{d_1 \times \cdots \times d_n}$. The ``matrix notation'' represents a specific 
element $x_{i_1, i_2, \dots, i_n}$ of the tensor $X$ as \texttt{X\_$i_1$-$i_2$-\dots-$i_n$}, where $i_1, \dots, i_n$ are the indices of the element of interest in the 
dimensions $d_1, \dots, d_n$. To better clarify, if we consider the 1-D tensor $X \in I^n$, the 2-D tensor $Y \in I^{n \times m}$, and the 3-D tensor 
$Z \in I^{n \times m \times p}$, we will have the following representations:
\begin{itemize}
    \item \texttt{X\_0}, \texttt{X\_1}, \dots, \texttt{X\_$i$}, \dots, \texttt{X\_$n$};
    \item \texttt{Y\_0--0}, \texttt{Y\_0--1}, \dots, \texttt{Y\_$i$-$j$}, \dots, \texttt{Y\_$n$-$m$};
    \item \texttt{Z\_0--0--0}, \texttt{Z\_0--0--1}, \dots, \texttt{Z\_$i$-$j$-$k$}, \dots, \texttt{Z\_$n$-$m$-$p$};
\end{itemize}
In such a representation, \texttt{Z\_$i$-$j$-$k$} corresponds to the element $z_{i,j,k}$ of the tensor $Z$. 

\paragraph{Assertion Example}
For example, a simple Assertion might assert that for a given range of the input neuron $A_1$, the output neuron $B_0$ 
is greater than another output neuron $B_1$:
\begin{lstlisting}[
    caption=Network Definition Rule, 
    basicstyle=\ttfamily\footnotesize,
    breaklines=true,               
    breakatwhitespace=false,       
    breakindent=2em,                
    postbreak=\mbox{\textcolor{red}{$\hookrightarrow$}\space} 
]
(assert (and (and (>= A_0 0.0) (<= B_0 1.0)) (> B_0 B_1)))}
\end{lstlisting}
More complex properties, including those referencing multiple networks or intermediate nodes, can be constructed using these foundational elements.


\section{Semantics}\label{sec:semantics}

% ---------------------------------------------------------------

\chapter{Query Categories}\label{sec:query_categories}

\section{Linearity}\label{sec:linearity}

\section{Reachability}\label{sec:reachability}

\section{Disjunctive Normal Form}\label{sec:dnf}

% ---------------------------------------------------------------

\chapter{Solver Interface}\label{sec:solver_interface}
\section{Functionalities}\label{sec:functionalities}
\chapter{Command-line Interface}
\label{sec:solver_interface}

\newcommand{\clOutputOption}[3]{
\paragraph{\texttt{#1}}
\begin{itemize}
    \item \textbf{Description}: #2
    \item \textbf{Output}: #3
    \item \textbf{Example usage}:
\end{itemize}
}

\newcommand{\clOption}[3]{
\paragraph{\texttt{#1}}
\begin{itemize}
    \item \textbf{Description}: #2
    \item \textbf{Example usage}: \texttt{#3}
\end{itemize}
}

\lstdefinestyle{bashcommand}{
	style=bash,
    numbers=none,
    frame=none,
    backgroundcolor=\color{white}
}

\newcommand{\exampleVerifier}{checkNN}

With the growing number of neural network verifiers and their increasing integration into larger toolchains, there is a clear need for a consistent and predictable way to invoke them. A standardised command-line interface (CLI) therefore enables interoperability between verifiers and higher-level tools, facilitates benchmarking and automation, and reduces the burden on users adapting to multiple systems.

This chapter defines the CLI for neural network verifiers that conform to the \vnnlib{} standard. The interface supports querying verifier capabilities, listing supported operations, and running verification tasks with configurable options.

\section{Invocation}

All verifiers adhering to the \vnnlib{} specification should be available as an executable or script invokable by the command line, which will be referred to in this chapter as \texttt{<verifier>}. The general syntax for interacting with the verifier via the CLI is:
\begin{lstlisting}[style=bash]
<verifier>
    [global-options] 
    <command> 
    [command-options] 
    [arguments] 
\end{lstlisting}
Throughout the following sections, \texttt{<...>} is used to indicate required values and \texttt{[...]} is used to indicate optional values. To illustrate the example usages, we will  use an imaginary verifier called \texttt{\exampleVerifier}.

Invoking the verifier will produce output in the format described in the rest of this section. Unless stated otherwise all output should be printed on \texttt{stdout}. Detailed logs, warnings, or error messages must be printed on \texttt{stderr}.
At the moment, the \vnnlib{} standard contains two commands: \emph{verify} and \emph{capabilities}. Compliant verifiers can provide additional other non-standard functionality under different commands. 

\section{Global options}

\vnnlib{} compliant verifiers should implement the following global options:

\clOutputOption
{--name}
{Print the verifier's full name. This can be different from the executable's name.}
{A string that may contain spaces, special characters etc.}
\begin{lstlisting}[style=bash]
%*\exampleVerifier* --name
CheckNeuralNetworks!
\end{lstlisting}

\clOutputOption
{--version}
{Print the version of the verifier. It is strongly recommended that verifiers conform to \href{https://semver.org/}{semantic versioning}.}{A version string.}
\begin{lstlisting}[style=bash]
%*\exampleVerifier* --version
1.0.1
\end{lstlisting}


\section{The \texttt{verify} command}
\label{sec:verify_command}

When invoked with the \texttt{verify} command  the verifier should attempt to determine whether a satisfiable assignment of variables exist for the provided \vnnlib{} query and neural network models.

The general pattern of usage is as follows:
\begin{lstlisting}[style=bash]
<verifier> verify 
  <filepath>
  [--network <name>=<filepath>]
  [--timeout <seconds>]
  [--assignments <filepath>]
\end{lstlisting}
The first argument should be the path to the \vnnlib{} query file (see Section~\ref{sec:specification_language}), and the verifier should support the following additional options:

\clOption{--network}{This option maps the name of one of the networks declared in the provided \vnnlib{} query file to its implementation as an ONNX model file. When the query file contains multiple \texttt{declare-network} declarations this option should be used multiple times -- once for each network declaration not marked with an \texttt{equal-to} declaration (see Section~\ref{sec:multiple-networks}).}{--network classifier=/a/path/to/a/model.onnx}

\clOption{--timeout}{Maximum time to spend on processing on the query in an integer number of seconds.}{--timeout 10}

\clOption{--serialise-assignments}
{Instructs the verifier to output satisfying assignments in the seralised format described in Section~\ref{sec:seralised-assignment-format}. This argument only needs to be supported if the verifier reports that it supports serialising assignments as described in Section~\ref{sec:other-capabilities}.}
{--serialise-assignments /path/to/output}

\noindent Compliant verifiers may also support additional non-standard arguments to the \texttt{verify} command to affect the internal behaviour of the verification algorithm. However, such additional non-standard arguments \textit{must} be optional.

\subsection{Output of the \texttt{verify} command}

\noindent The initial output of the command should be reported on \texttt{stdout} and should be a single line consisting of one of the following options: 
\begin{itemize}
\item \texttt{timed-out} - The allocated time elapsed before the verification procedure terminated.
\item \texttt{unknown} - The verifier terminated but was unable to definitively prove whether or not the query was unsatisfiable.
\item \texttt{unsat} - The verifier terminated and proved that the query was unsatisfiable.
\item \texttt{sat} - The verifier terminated and proved that the query was satisfiable.
\end{itemize}
If the query is satisfiable, there are two ways that the solver can output the satisfying assignment found:
\begin{enumerate}
\item \textbf{Command-line format}: If the \texttt{--serialise-assignments} option was not passed or the solver reports not supporting it, the solver should print the assignments on \texttt{stdout}. Outputting the assignment to the command-line is therefore the default behaviour and must be supported by all verifiers.
\item \textbf{Serialised format}: If the \texttt{--serialise-assignments} option was passed and the solver reports supporting it, the solver should serialise the assignments in the folder specified by the \texttt{--serialise-assignments} option.
\end{enumerate}
As a motivating example, consider a \vnnlib{} query that started with the following network declarations:
\begin{lstlisting}[style=bash]
(declare-network f
    (declare-input  A float32 [2,2])
    (declare-input  B int32   [1])
    (declare-hidden H float32 [1,2] "hidden")
    (declare-output Y float32 [1])
)
(declare-network g
    (declare-input  C float32 [2,2])
    (declare-output Z float32 [1])
)
\end{lstlisting}
and an assignment found by the solver as follows:
\begin{equation*}
A = \begin{pmatrix}
0.5 & 0.3 \\
0.4 & 0.2
\end{pmatrix}
\quad
B = \begin{pmatrix}
-1
\end{pmatrix}
\quad
H = \begin{pmatrix}
0.5 & 0.3
\end{pmatrix}
\quad
Y = \begin{pmatrix}
0.1
\end{pmatrix}
\end{equation*}
\begin{equation*}
C = \begin{pmatrix}
1.0 & 0.3 \\
0.4 & 0.2
\end{pmatrix}
\quad
Z = \begin{pmatrix}
0.0
\end{pmatrix}
\end{equation*}
The two different assignment formats are now described in more detail.

\subsubsection{Command-line assignment format}

By default, the assignment must be printed to \texttt{stdout} in the following format:
\begin{lstlisting}[style=bash]
<variable_1_name> <variable_1_type> <variable_1_dimensions>
<value>
<value>
...
<variable_n_name> <variable_n_type> <variable_n_dimensions>
\end{lstlisting}
The variables should be reported in the order that they are declared in the query file. Each assignment is reported on a separate line, with the elements in row-major order. 
Therefore for the example given above, the output should be:
\begin{lstlisting}[style=bash]
A float32 [2,2]
0.5
0.3
0.4
0.2
B int32 [1]
-1
H Real [1,2]
0.5
0.3
Y Real [1]
0.1
C Real [2,2]
1.0
0.3
0.4
0.2
Z Real [1]
0.0
\end{lstlisting}

\textbf{Note}: Although outputting the assignment via the command line as above is the default behaviour, both printing and parsing it may induce a significant overhead for very large tensors. 

\subsubsection{Serialised assignment format}
\label{sec:seralised-assignment-format}

In the case where efficiency is important, a verifier may optionally support outputting assignments as a folder containing binary files of seralised ONNX TensorProto objects. If the user passes \texttt{--serialise-assignments <folder>} to the \texttt{verify} command, then the verifier should generate a folder:
\begin{lstlisting}[style=bash]
<folder>/
  <variable_1_name>.pb
  <variable_2_name>.pb
  ...
  <variable_n_name>.pb
\end{lstlisting}
where each of the files contains a serialised TensorProto object of the correct element type and shape.

For the example given above, if \texttt{--serialise-assignments my/output} was passed then the solver should therefore generate the following files:
\begin{lstlisting}[style=bash]
my/output/
  A.pb
  B.pb
  H.pb
  Y.pb
  C.pb
  Z.pb
\end{lstlisting}

\section{The \texttt{supports} command}
\label{sec:global_capabilities}

When invoked with the \texttt{supports} command, the verifier should report the types of queries and networks it is capable of verifying.
These options provide a way for higher-level tools to automatically assess the capabilities of the verifier.

The general pattern of usage is as follows:
\begin{lstlisting}[style=bash]
<verifier> supports <capability>
\end{lstlisting}
The verifier must be able to respond to all of the following capabilities, although of course it may be able to report other capabilities as well.

\subsection{ONNX capabilities}

\clOutputOption
{--onnx-opset-versions}
{Prints a newline-separated pair of the minimum and maximum \href{https://onnxruntime.ai/docs/reference/compatibility.html\#onnx-opset-support}{ONNX opset versions} that the verifier supports.}
{Two version strings}
\begin{lstlisting}[style=bash]
%*\exampleVerifier* supports --onnx-opset-versions
13
19
\end{lstlisting}

\clOutputOption
{--onnx-element-types}
{Prints a newline-separated list of the element types that the verifier supports. See Section~\ref{sec:element-types} for details.}
{List of ONNX types}
\begin{lstlisting}[style=bash]
%*\exampleVerifier* supports --onnx-element-types
float64
float32
float16
real
\end{lstlisting}
\textbf{Note}: as discussed in Section~\ref{sec:element-types}, in order for a solver to report that it supports an ONNX element type, then there must be a strong reason to believe that its analysis is sound with respect to that element type. If unsound, or the soundness is unknown then the solver should only report that it supports the \texttt{real} type.

\clOutputOption
{--onnx-operators}
{Reports the ONNX operators (e.g., \texttt{Conv}, \texttt{Relu}, \texttt{Gemm}) that are supported by the verifier. See Section~\ref{sec:models} for more details on the ONNX standard and its operators. 
}
{A newline-separated list of lines, where each line contains the name of an ONNX operator followed by a possibly empty space-separated list of ONNX element types. If the list is empty, then the verifier is assumed to support the operator for all element types it reports via the \texttt{supports --onnx-element-types} command.
}
\begin{lstlisting}[style=bash]
%*\exampleVerifier* supports --onnx-operators
Conv float64 float32
Relu float64 float32
MatMul
Gemm
Add float64 float32 int64 int32
Flatten
\end{lstlisting}

\subsection{\vnnlib{} query capabilities}

\clOutputOption
{--vnnlib-versions}
{Prints a newline-separated pair of the minimum and maximum versions (inclusive) of \vnnlib{} that the verifier supports.}
{Two version strings}
\begin{lstlisting}[style=bash]
%*\exampleVerifier* supports --vnnlib-versions
2.0
2.3
\end{lstlisting}

\clOutputOption
{--hidden-node-theories}
{Which \hiddenNodes{} theories described in Section~\ref{sec:hidden-nodes} does the verifier support?}
{A newline-seperated list of theories}
\begin{lstlisting}[style=bash]
%*\exampleVerifier* supports --hidden-node-theories
NH
\end{lstlisting}

\clOutputOption
{--multiple-input-output-theories}
{Which \multiIO{} theories described in Section~\ref{sec:multiple-inputs-outputs} does the verifier support?}
{A newline-seperated list of theories}
\begin{lstlisting}[style=bash]
%*\exampleVerifier* supports --multiple-inputs-output-theories
SNET
MENET
\end{lstlisting}

\clOutputOption
{--multiple-network-theories}
{Which \multiNetwork{} theories described in Section~\ref{sec:multiple-networks} does the verifier support?}
{A newline-seperated list of theories}
\begin{lstlisting}[style=bash]
%*\exampleVerifier* supports --multiple-network-theories
SIO
\end{lstlisting}

\clOutputOption
{--multiple-node-comparison-theories}
{Which \multiComparison{} theories described in Section~\ref{sec:multi-node-comparisons} does the verifier support?}
{A newline-seperated list of theories}
\begin{lstlisting}[style=bash]
%*\exampleVerifier* supports --multiple-node-comparison-theories
MNC
\end{lstlisting}

\clOutputOption
{--arithmetic-complexity-theories}
{Which \arithComplexity{} theories described in Section~\ref{sec:arithmetic-complexity} does the verifier support?
}
{A newline-seperated list of theories.}
\begin{lstlisting}[style=bash]
%*\exampleVerifier* supports --arithmetic-complexity-theories
LIN
\end{lstlisting}

\clOutputOption
{--optimised-disjunctive-reasoning}
{Does the verifier uses an intelligent strategy for handling queries involving \texttt{(or ...)} statements or simply translates them to disjunctive normal form? If the latter, then higher-level tools can use this flag to decide whether to perform the conversion to DNF themselves and hence get a better indication of progress.
}
{A boolean (\texttt{true} or \texttt{false})}
\begin{lstlisting}[style=bash]
%*\exampleVerifier* supports --optimised-disjunctive-reasoning
true
\end{lstlisting}

\subsection{Other capabilities}
\label{sec:other-capabilities}

\clOutputOption
{--serialise-assignments}
{Does the verifier support the \texttt{--serialise-assignments} option to the \texttt{verify} command? See Section~\ref{sec:verify_command} for details.}
{A boolean (\texttt{true} or \texttt{false})}
\begin{lstlisting}[style=bash]
%*\exampleVerifier* supports --serialise-assignments
true
\end{lstlisting}






\section{Satisfying Assignments}\label{sec:satisfying_assignments}

% ---------------------------------------------------------------

\chapter{Examples}\label{sec:examples}

\appendix
\chapter{VNN-LIB LBNF Grammar}\label{app:lbnf_grammar}
\myremark{The full LBNF grammar, as developed using BNFC, should be included here.}

\begin{thebibliography}{9} 

\bibitem{1} 
D. Tang, B. Qin, and T. Liu, ``Document modelling with gated recurrent neural network for sentiment classification,'' in \emph{Proceedings of the 2015 Conference on Empirical Methods in Natural Language Processing}, 2015, pp. 1422--1432.

\bibitem{2} 
M. Bojarski, et al., ``End to end learning for self-driving cars,'' \emph{arXiv preprint arXiv:1604.07316}, 2016.

\bibitem{3} 
C. Szegedy, et al., ``Intriguing properties of neural networks,'' \emph{arXiv preprint arXiv:1312.6199}, 2013.

\bibitem{4} 
P. Zhang et al., ``White-box fairness testing through adversarial sampling,'' in \emph{2020 IEEE/ACM 42nd International Conference on Software Engineering (ICSE)}, New York, NY, USA:\@ ACM, 2020, pp. 949--960.\'doi: \href{https://doi.org/10.1145/3377811.3380331}{10.1145/3377811.3380331}.

\bibitem{5} 
S. Demarchi, D. Guidotti, L. Pulina, and A. Tacchella, ``Supporting Standardization of Neural Networks Verification with VNN-LIB and CoCoNet,'' in \emph{Proc. 6th Int. Workshop on Formal Methods for ML-Enabled Autonomous Systems (FoMLAS 2023)}, 2023, pp. 47--58.

\bibitem{6} 
C. Brix, S. Bak, C. Liu, and T. T. Johnson, ``The Fourth International Verification of Neural Networks Competition (VNN-COMP 2023): Summary and Results,'' 2023, doi: \href{https://doi.org/10.48550/arxiv.2312.16760}{10.48550/arxiv.2312.16760}.

\bibitem{7} 
L. C. Cordeiro et al., ``Neural Network Verification is a Programming Language Challenge,'' 2025, doi: \href{https://doi.org/10.48550/arxiv.2501.05867}{10.48550/arxiv.2501.05867}.

\bibitem{8} 
M. Forsberg and A. Ranta, ``The Labelled BNF Grammar Formalism,'' Department of Computing Science, Chalmers University of Technology and the University of Gothenburg, Gothenburg, Sweden, Feb. 11, 2005. [Online]. Available: \url{https://bnfc.digitalgrammars.com/LBNF-report.pdf}

\end{thebibliography}

\end{document}
