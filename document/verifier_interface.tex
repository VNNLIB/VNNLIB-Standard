% Define some colors for listings
\definecolor{codegreen}{rgb}{0,0.6,0}
\definecolor{codegray}{rgb}{0.5,0.5,0.5}
\definecolor{codepurple}{rgb}{0.58,0,0.82}
\definecolor{backcolour}{rgb}{0.95,0.95,0.92}
\definecolor{keywordblue}{rgb}{0.13,0.13,1}
\definecolor{stringred}{rgb}{0.8,0,0}

% Listings style for Bash/CLI commands
\lstdefinestyle{bash}{
    backgroundcolor=\color{backcolour},   
    commentstyle=\color{codegreen},
    keywordstyle=\color{keywordblue}\bfseries,
    numberstyle=\tiny\color{codegray},
    stringstyle=\color{stringred},
    basicstyle=\ttfamily\footnotesize,
    breakatwhitespace=false,         
    breaklines=true,                 
    captionpos=b,                    
    keepspaces=true,                 
    numbers=left,                    
    numbersep=5pt,                  
    showspaces=false,                
    showstringspaces=false,
    showtabs=false,                  
    tabsize=2,
    frame=lines,
    xleftmargin=2em,
    framexleftmargin=1.5em,
    escapeinside={\%*}{*} 
}

\section{Introduction}
This chapter specifies the command-line interface (CLI) for neural network verifiers compliant with the VNNLIB-2.0 standard. The goal is to provide a consistent and 
predictable interface for users and automated tools to interact with different verifiers. The CLI supports querying verifier capabilities, listing supported operations, 
and invoking the verification process with various configurations.

All verifiers adhering to this specification should expose an executable, referred to in this document as \texttt{vnn\_verifier}.

\section{General Usage}
The general syntax for interacting with the verifier via the CLI is:
\begin{lstlisting}[style=bash, numbers=none, frame=none, backgroundcolor=\color{white}]
vnn_verifier [global-options] <command> [command-options] 
	[arguments] \end{lstlisting}

\section{Global Options}
These options are applicable to the main \texttt{vnn\_verifier} executable and potentially to all commands.
\begin{itemize}
    \item \texttt{--help}, \texttt{-h}: Display help information for the CLI or for a specific command if used like \texttt{vnn\_verifier <command> --help}.
    \item \texttt{--version}, \texttt{-v}: Display the verifier's name and version.
\end{itemize}

\section{Commands}

\subsection{\texttt{list-onnx-ops}}
\begin{itemize}
    \item \textbf{Description}: Prints a newline-separated list of ONNX operator types (e.g., ``Conv'', ``Relu'', ``Gemm'') that are supported by the verifier. 
	This helps users determine if their ONNX model is compatible with the verifier.
    \item \textbf{Usage}:
    \begin{lstlisting}[style=bash]
vnn_verifier list-onnx-ops \end{lstlisting}
    \item \textbf{Example Output}:
    \begin{lstlisting}[style=bash, numbers=none, frame=none, backgroundcolor=\color{white}]
Conv
Relu
MatMul
Gemm
Add
Flatten
...
    \end{lstlisting}
\end{itemize}

\subsection{\texttt{capabilities}}
\begin{itemize}
    \item \textbf{Description}: Displays a comprehensive summary of the verifier's capabilities, features, and limitations in a structured format.
    \item \textbf{Usage}:
    \begin{lstlisting}[style=bash]
vnn_verifier capabilities [--format <text|json|yaml>] \end{lstlisting}
    \item \textbf{Options}:
    \begin{itemize}
        \item \texttt{--format <format>}: Specifies the output format. Supported formats are \texttt{text} (human-readable, default), \texttt{json}, and \texttt{yaml} (machine-readable).
    \end{itemize}
    \item \textbf{Example JSON Output} (\texttt{vnn\_verifier capabilities --format json}):
    \begin{lstlisting}[style=bash, caption=Example JSON output for capabilities command]
{
  "verifier_name": "ExampleVerifier",
  "verifier_version": "1.2.3",
  "description": "ExampleVerifier is a neural network verifier that supports VNNLIB 2.0.",
  "vnnlib_support": {
		"multiple_networks": true,
		"multiple_io": true,
		"hidden_nodes": false
  },
  "assertion_support": {
		"disjunction": true,
		"conjunction": true,
		"required_normal_form": "DNF",
		%* (May be "DNF", "CNF", "NNF", or "None") *
		"boolean_operators": ["<=", ">=", "==", "!="],
		"arithmetic_operators": ["+", "-", "*"],
		"arithmetic_complexity": "Linear"
		%* (May be "Linear", "Polynomial", "GeneralNonLinear") *
  },
  "query_support": {
    	"query_format": ["VNNLIB", "Marabou"]
  },
  "reachability_constraints":{
		"reachability_based": true,
		"input_domains": ["Box", "Zonotope", "Polytope"],
		"output_constraints": ["Linear", "Polyhedral"]
  }
  "model_support": {
		"max_layers": 128,
		"max_nodes_per_layer": 1024,
		"supported_types": ["float16"],
		"onnx_operators": ["Conv", "Relu", "Gemm", "MatMul"]
  },
  "artifact_support": {
		"counterexample": true,
		"proof_certificate": true,
		"counterexample_format": ["VNNLIB", "Marabou"],
		"proof_certificate_format": ["VNNLIB", "Marabou"]
  }
}
    \end{lstlisting}
\end{itemize}

\subsection{\texttt{verify}}
\begin{itemize}
    \item \textbf{Description}: Invokes the verification process on a given neural network model and a VNNLIB property specification.
    \item \textbf{Usage}:
    \begin{lstlisting}[style=bash]
vnn_verifier verify
	 --network <name1>:<net1_path> 
	[--network <name2>:<net2_path> ...] 
	 --property <vnnlib_filepath> [options] \end{lstlisting}
    \item \textbf{Options}:
    \begin{longtable}{@{}>{\raggedright\arraybackslash\ttfamily}p{0.28\textwidth}% Option column
                    >{\raggedright\arraybackslash\ttfamily}p{0.12\textwidth}% Shorthand column
                    >{\raggedright\arraybackslash\ttfamily}p{0.20\textwidth}% Argument column
                    >{\raggedright\arraybackslash}p{0.32\textwidth}@{}}% Description column (normal font for readability)
        \toprule
        Option & Shorthand & Argument & Description \\
        \midrule
        \bottomrule
        \texttt{--network} & \texttt{-n} & \texttt{<name>:<path>} & \textbf{(Required)} Identifier for the network and path to the ONNX model file. \\
        \texttt{--property} & \texttt{-p} & \texttt{<path>} & \textbf{(Required)} Path to the VNNLIB 2.0 property file. \\
        \texttt{--timeout} & \texttt{-t} & \texttt{<seconds>} & Maximum verification time in integer seconds. \\
        \texttt{--counterexample} & \texttt{-cex} & \texttt{<path>} & Path to save the counterexample if property is violated. \\
        \texttt{--proof} & \texttt{-pf} & \texttt{<path>} & Path to save the proof certificate if property holds. \\
    \end{longtable}
    \item \textbf{Standard Output/Error}:
    \begin{itemize}
        \item The primary result is printed to \texttt{stdout} (e.g., \texttt{SAT}, \texttt{UNSAT}, \texttt{TIMEOUT}, \texttt{UNKNOWN}, \texttt{ERROR:\@<message>}).
        \item If \texttt{UNSAT} and a counterexample is generated, it will be printed to \texttt{stdout} or saved to the specified file.
        \item Detailed logs or error messages are sent to \texttt{stderr}.
    \end{itemize}
    \item \textbf{Exit Codes}:
    \begin{itemize}
        \item \texttt{0}: Property is satisfied (SAT).
        \item \texttt{1}: Property is violated (UNSAT).
        \item \texttt{2}: Verification timed out (TIMEOUT).
        \item \texttt{3}: Result of verification is unknown (UNKNOWN).
        \item \texttt{4}: An error occurred during setup or verification (e.g., file not found, unsupported ONNX operator, malformed VNNLIB specification, internal verifier error).
        \item \texttt{5}: Invalid command-line arguments provided to the verifier.
    \end{itemize}
\end{itemize}

\subsection{\texttt{supports}}
\begin{itemize}
    \item \textbf{Description}: Checks if the verifier supports a specific named feature. This provides a quick way to query individual capabilities without parsing the full \texttt{capabilities} output.
    \item \textbf{Usage}:
    \begin{lstlisting}[style=bash]
vnn_verifier supports --feature <feature_name> \end{lstlisting}
    \item \textbf{Required Options}:
    \begin{itemize}
        \item \texttt{--feature <feature\_name>}: The name of the feature to query.
    \end{itemize}
    \item \textbf{Output}:
    \begin{itemize}
        \item Prints ``true'' to \texttt{stdout} if the feature is supported.
        \item Prints ``false'' to \texttt{stdout} if the feature is not supported or the feature name is unrecognized.
    \end{itemize}
    \item \textbf{Exit Codes}:
    \begin{itemize}
        \item \texttt{0}: Feature is supported (and ``true'' is printed).
        \item \texttt{1}: Feature is not supported (and ``false'' is printed).
        \item \texttt{2}: Invalid or unrecognized \texttt{<feature\_name>}.
    \end{itemize}
    \item \textbf{Supported \texttt{<feature\_name>} values} (these should align with keys in the \texttt{capabilities} JSON output where applicable):
    \begin{itemize}
        \item \texttt{multiple-networks}
        \item \texttt{multiple-io}
        \item \texttt{hidden-nodes}
    \end{itemize}
    Note: The above provides common examples, but is incomplete
\end{itemize}