% Define some colors for listings
\definecolor{codegreen}{rgb}{0,0.6,0}
\definecolor{codegray}{rgb}{0.5,0.5,0.5}
\definecolor{codepurple}{rgb}{0.58,0,0.82}
\definecolor{backcolour}{rgb}{0.95,0.95,0.92}
\definecolor{keywordblue}{rgb}{0.13,0.13,1}
\definecolor{stringred}{rgb}{0.8,0,0}

% Listings style for Bash/CLI commands
\lstdefinestyle{bash}{
    backgroundcolor=\color{backcolour},   
    commentstyle=\color{codegreen},
    keywordstyle=\color{keywordblue}\bfseries,
    numberstyle=\tiny\color{codegray},
    stringstyle=\color{stringred},
    basicstyle=\ttfamily\footnotesize,
    breakatwhitespace=false,         
    breaklines=true,                 
    captionpos=b,                    
    keepspaces=true,                 
    numbers=left,                    
    numbersep=5pt,                  
    showspaces=false,                
    showstringspaces=false,
    showtabs=false,                  
    tabsize=2,
    frame=lines,
    xleftmargin=2em,
    framexleftmargin=1.5em,
    escapeinside={\%*}{*} 
}

\section{Introduction}
This chapter specifies the command-line interface (CLI) for neural network verifiers compliant with the VNNLIB-2.0 standard. The goal is to provide a consistent and 
predictable interface for users and automated tools to interact with different verifiers. The CLI supports querying verifier capabilities, listing supported operations, 
and invoking the verification process with various configurations.

All verifiers adhering to this specification should expose an executable, referred to in this document as \texttt{vnn\_verifier}.

\section{General Usage}
The general syntax for interacting with the verifier via the CLI is:
\begin{lstlisting}[style=bash, numbers=none, frame=none, backgroundcolor=\color{white}]
vnn_verifier [global-options] <command> [command-options] 
	[arguments] \end{lstlisting}

\section{Global Options}
These options are applicable to the main \texttt{vnn\_verifier} executable and potentially to all commands.
\begin{itemize}
    \item \texttt{--help}, \texttt{-h}: Display help information for the CLI or for a specific command if used like \texttt{vnn\_verifier <command> --help}.
    \item \texttt{--name}, \texttt{-v}: Display the verifier's name.
    \item \texttt{--version}, \texttt{-V}: Display the verifier's version.
\end{itemize}

\section{Commands}

\subsection{\texttt{verify}}
\begin{itemize}
    \item \textbf{Description}: Invokes the verification process on a given neural network model and a VNNLIB property specification.
    \item \textbf{Usage}:
    \begin{lstlisting}[style=bash]
vnn_verifier verify
	 --network <name1>:<net1_path> 
	[--network <name2>:<net2_path> ...] 
	 --property <vnnlib_filepath> [options] \end{lstlisting}
    \item \textbf{Options}:
    \begin{longtable}{@{}>{\raggedright\arraybackslash\ttfamily}p{0.28\textwidth}% Option column
                    >{\raggedright\arraybackslash\ttfamily}p{0.12\textwidth}% Shorthand column
                    >{\raggedright\arraybackslash\ttfamily}p{0.20\textwidth}% Argument column
                    >{\raggedright\arraybackslash}p{0.32\textwidth}@{}}% Description column (normal font for readability)
        \toprule
        Option & Shorthand & Argument & Description \\
        \midrule
        \bottomrule
        \texttt{--network} & \texttt{-n} & \texttt{<name>:<path>} & \textbf{(Required)} Identifier for the network and path to the ONNX model file. \\
        \texttt{--property} & \texttt{-p} & \texttt{<path>} & \textbf{(Required)} Path to the VNNLIB 2.0 property file. \\
        \texttt{--timeout} & \texttt{-t} & \texttt{<seconds>} & Maximum verification time in integer seconds. \\
        \texttt{--dataset} & \texttt{-d} & \texttt{<path>} & Path to a dataset file (optional, for models that require input data).\\
        \texttt{--output} & \texttt{-o} & \texttt{<path>} & Path to save the satisfying assignment (if applicable). \\
    \end{longtable}
    \item \textbf{Standard Output/Error}:
    \begin{itemize}
        \item The primary result is printed to \texttt{stdout}: \texttt{SAT}, \texttt{UNSAT}, \texttt{TIMEOUT}, \texttt{UNKNOWN}.
        \item If \texttt{SAT}, the generated satisfying assignment will be printed to \texttt{stdout} or saved to the specified file.
        \item Detailed logs or error messages are sent to \texttt{stderr}.
    \end{itemize}
\end{itemize}

\myremark{The \texttt{<name>} in the \texttt{--network} option is a user-defined identifier for the network, which is used in the VNNLIB property file to refer to this specific network.}

\section{Granular Capability Options}
This section details additional global options for querying individual verifier capabilities. These options provide a convenient way to check for specific features of the verifier.

\subsection{\texttt{supports-multiple-networks}}
\begin{itemize}
    \item \textbf{Shorthand}: \texttt{mn}
    \item \textbf{Description}: Checks if the verifier supports properties defined over multiple neural networks. 
    \item \textbf{Usage}: \texttt{vnn\_verifier --supports-multiple-networks}
    \item \textbf{Output}: Prints "true" or "false" to stdout.
\end{itemize}

\subsection{\texttt{supports-multiple-input-output}}
\begin{itemize}
    \item \textbf{Shorthand}: \texttt{mio}
    \item \textbf{Description}: Checks if the verifier supports models with multiple input and/or output tensors. 
    \item \textbf{Usage}: \texttt{vnn\_verifier --supports-multiple-io}
    \item \textbf{Output}: Prints "true" or "false" to stdout.
\end{itemize}

\subsection{\texttt{supports-hidden-nodes}}
\begin{itemize}
    \item \textbf{Shorthand}: \texttt{hn}
    \item \textbf{Description}: Checks if the verifier supports referencing hidden layer nodes in VNNLIB properties. 
    \item \textbf{Usage}: \texttt{vnn\_verifier --supports-hidden-nodes}
    \item \textbf{Output}: Prints "true" or "false" to stdout.
\end{itemize}

\subsection{\texttt{requires-dnf}}
\begin{itemize}
    \item \textbf{Shorthand}: \texttt{dnf}
    \item \textbf{Description}: Checks if the verifier requires assertions to be in Disjunctive Normal Form (DNF). 
    \item \textbf{Usage}: \texttt{vnn\_verifier --requires-dnf}
    \item \textbf{Output}: Prints "true" or "false" to stdout.
\end{itemize}

\subsection{\texttt{supports-strict-comparison}}
\begin{itemize}
    \item \textbf{Shorthand}: \texttt{sc}
    \item \textbf{Description}: Checks if the verifier supports strict inequalities (\texttt{<}, \texttt{>}). 
    \item \textbf{Usage}: \texttt{vnn\_verifier --supports-strict-comparison}
    \item \textbf{Output}: Prints "true" or "false" to stdout.
\end{itemize}

\subsection{\texttt{requires-linear-complexity}}
\begin{itemize}
    \item \textbf{Shorthand}: \texttt{lc}
    \item \textbf{Description}: Checks if the verifier's supported assertions are limited to linear arithmetic. 
    \item \textbf{Usage}: \texttt{vnn\_verifier --requires-linear-complexity}
    \item \textbf{Output}: Prints "true" or "false" to stdout.
\end{itemize}

\subsection{\texttt{is-reachability-based}}
\begin{itemize}
    \item \textbf{Shorthand}: \texttt{rb}
    \item \textbf{Description}: Checks if the verifier is a reachability-based tool. 
    \item \textbf{Usage}: \texttt{vnn\_verifier --is-reachability-based}
    \item \textbf{Output}: Prints "true" or "false" to stdout.
\end{itemize}

\subsection{\texttt{supported-domains}}
\begin{itemize}
    \item \textbf{Shorthand}: \texttt{sd}
    \item \textbf{Description}: Prints a newline-separated list of the abstract input domain representations supported by the verifier. 
    \item \textbf{Usage}: \texttt{vnn\_verifier --list-input-domains}
    \item \textbf{Example Output}:
    \begin{lstlisting}[style=bash, numbers=none, frame=none, backgroundcolor=\color{white}]
Box
Zonotope
Polytope
...
    \end{lstlisting}
\end{itemize}

\subsection{\texttt{supported-types}}
\begin{itemize}
    \item \textbf{Shorthand}: \texttt{st}
    \item \textbf{Description}: Prints a newline-separated list of supported ONNX data types.
    \item \textbf{Usage}: \texttt{vnn\_verifier --list-supported-types}
    \item \textbf{Example Output}:
    \begin{lstlisting}[style=bash, numbers=none, frame=none, backgroundcolor=\color{white}]
float16
float32
...
    \end{lstlisting}
\end{itemize}

\subsection{\texttt{supported-onnx-ops}}
\begin{itemize}
    \item \textbf{Shorthand}: \texttt{so}
    \item \textbf{Description}: Prints a newline-separated list of ONNX operator types (e.g., ``Conv'', ``Relu'', ``Gemm'') that are supported by the verifier. 
	This helps users determine if their ONNX model is compatible with the verifier.
    \item \textbf{Usage}: \texttt{vnn\_verifier --list-onnx-ops}
    \item \textbf{Example Output}:
    \begin{lstlisting}[style=bash, numbers=none, frame=none, backgroundcolor=\color{white}]
Conv
Relu
MatMul
Gemm
Add
Flatten
...
    \end{lstlisting}
\end{itemize}