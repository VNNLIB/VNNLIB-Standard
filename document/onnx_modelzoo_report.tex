Here we describe the operators that are officially supported, i.e.,
those operators supplied by the ONNX model 
zoo~\footnote{https://github.com/onnx/models} that allow to represent
the majority of the models in the zoo limiting as much as possible the
variety. We consider the model zoo as representative enough for the 
kind of model architectures, and operators, that are commonly used in
the Machine Learning community.

The following operators cover almost every benchmark provided in the
VNN-COMP repositories for sequential networks; other kinds of networks
(ResNet, Recurrent, etc.) are often based on "exotic" and, in general,
peculiar operators that do not lie in this list.

\myremark{While the standard supports the operators, it is strongly 
	unadvised to include pre-processing in the benchmark model, e.g., 
	normalization and flattening, since the properties should match
	the first node with the same input dimension.}

\begin{itemize}
	\item \emph{Add (Add)} operator performs the element-wise sum of
		a tensor and a scalar. We strongly encourage to use the 
		\textit{Gemm} operator when paired with \textit{MatMul}.
	
	\item \emph{AveragePool (Average Pooling)} operator
	  supports downsampling with averaging.
	
	\item \emph{BatchNormalization (Batch
	  Normalization)} operator supports adjusting and scaling the
	  activations functions, and it is expressive enough to represent
	  general batch normalization.
	  
	\item \emph{Concat (Concatenation)} operator concatenates a list
		of tensors into a single tensor, with the same shape except for
		the axis to concatenate on.	
	
	\item \emph{Conv (Convolutional)} operator supports
	  all the attributes to encode a generic convolutional layer.
	
	\item \emph{Dropout (Dropout)} operator supports
	  random dropping of units (during training). This operator should not
	  appear on trained models.  
	  
	\item \emph{Flatten (Flatten)} this operator converts multidimensional
	  arrays (tensors) to single dimensional ones; it is used instead of
	  \emph{Reshape} in some of the models in the zoo.
	  
	\item \emph{Gemm (General Matrix Multiplication)}
	  operator encodes matrix multiplication possibly with a scalar
	  coefficient and the addition of another matrix; as such \emph{Gemm}
	  can encode fully connected layers in neural networks.
	
	\item \emph{LRN (Local Response Normalization)}
	  operator supports normalization over local input regions; it is uatilized
	  in Alexnet and derived networks.
	  
	\item \emph{MatMul (Matrix Multiplication)} operator performs a
		numPy-like matrix multiplication. We strongly encourage to use
		the \textit{Gemm} operator.
	  
	\item \emph{MaxPool (Maximum pooling)} operator supports
	  downsampling with maximization.
	
	\item \emph{ReLU (Rectified Linear Unit)} operator
	  encodes the corresponding activation function $\sigma(x) = \max(0, x)$.
	  
	\item \emph{Reshape (Reshape)} operator supports
	  reshaping of the tensor's dimensions.
	  
	\item \emph{Sigmoid (Logistic Unit)} operator
	  encodes the corresponding activation function $\sigma(x) =
	  \frac{1}{1 + e^{-x}}$.
	  
	\item \emph{SoftMax (Softmax Unit)} operator transforms
	  vectors into probabilities, e.g., for selecting among different
	  classes and it is commonly utilized in state of the art
	  networks.
	  
	\item \emph{Sub (Sub)} operator performs element-wise binary subtraction
		between two tensors.
	
	\item \emph{Unsqueeze (Unsqueeze)} operator removes dimensions of size
	  1 from tensors, and it is utilized, e.g., in \emph{Densenet} and
	  \emph{Inception2}.
\end{itemize}

%% At a high level, networks can be seen as functions 
%% $\nu : I^n \to O^m$, mapping an $n$-dimensional \emph{input domain}
%% $I^n$ ($n > 0$) to a $m$-dimensional \emph{output domain} $O^m$ ($m
%% >0$). We argue that this representation captures most cases of practical
%% interest.
%% For instance, a network computing an approximation
%% of some function $f: \mathbb{R}^n \to \mathbb{R}$ would have $I = O =
%% \mathbb{R}$, whereas a network classifying 8-bit images of size $h \times v$ in
%%  two classes would be defined as ${\nu: \{0,\ldots,255\}^{h \cdot v}
%%    \to \{0, 1\}}$ with $I=\{0, \ldots, 255\}$ and $O = \{0,1\}$.


