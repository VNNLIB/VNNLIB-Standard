% Define some colors for listings
\definecolor{terminalbg}{rgb}{0.1,0.1,0.1}     % Dark background
\definecolor{terminalfg}{rgb}{0.9,0.9,0.9}     % Light foreground
\definecolor{terminalcomment}{rgb}{0.4,0.8,0.4} % Green for comments
\definecolor{terminalkeyword}{rgb}{0.4,0.6,1.0} % Blue-ish keywords
\definecolor{terminalstring}{rgb}{1.0,0.6,0.6}  % Light red strings
\definecolor{terminalnumber}{rgb}{1.0,0.8,0.4}  % Orange numbers

% Listings style for Bash/CLI commands
\lstdefinestyle{lbnf}{
    backgroundcolor=\color{terminalbg},
    basicstyle=\ttfamily\footnotesize\color{terminalfg},
    commentstyle=\color{terminalcomment}\itshape,
    keywordstyle=\color{terminalkeyword}\bfseries,
    stringstyle=\color{terminalstring},
    numberstyle=\tiny\color{terminalnumber},
    breakatwhitespace=false,
    breaklines=true,
    captionpos=b,
    keepspaces=true,
    numbers=left,
    numbersep=5pt,
    showspaces=false,
    showstringspaces=false,
    showtabs=false,
    tabsize=2,
    frame=lines,
    xleftmargin=2em,
    framexleftmargin=1.5em,
    escapeinside={\%*}{*}
}


The syntax of VNN-LIB 2.0 is formally defined using Labelled Backus-Naur Form (LBNF)\cite{8}. LBNF is a variant of BNF that allows for 
annotations (labels) on productions, facilitating the automatic generation of abstract syntax trees, parsers, and other language processing tools. 
This formal grammar provides a rigorous foundation for the language, eliminating ambiguities present in previous versions and ensuring consistent 
parsing across different tools.

The full LBNF grammar for VNN-LIB 2.0 is provided in the Appendix. The following subsections highlight key syntactic constructs introduced 
or modified in VNN-LIB 2.0, simplified for readability.

\subsection{Network Definitions}
VNN-LIB 2.0 supports the definition of one or more neural networks within a single specification file. This is crucial for properties involving 
multiple networks, such as checking equivalence or performing compositional verification.

A network definition is introduced by the keyword \texttt{declare-network}, followed by a user-defined variable name for the network, and then its 
associated input, intermediate (hidden), and output variable declarations, all enclosed in parentheses.

The LBNF rule can be generally represented as:
\begin{lstlisting} [
	caption=Network Definition Rule, 
	style=lbnf
]
NetworkDefinition ::= "(declare-network" VariableName InputDefinition+ IntermediateDefinition* OutputDefinition+ ")" ;
\end{lstlisting}
Here, \texttt{VariableName} is an identifier for the network.\texttt{InputDefinition+} indicates one or more input definitions, \texttt{IntermediateDefinition*} 
indicates zero or more intermediate (hidden) node definitions, and \texttt{OutputDefinition+} indicates one or more output definitions. These components are detailed below.

\subsection{Input and Output Variable Declarations}
In VNN-LIB 1.0, input and output variables were implicitly defined by the first and last nodes of the ONNX model, respectively. However, this approach lacked flexibility and expressiveness,
especially for networks with multiple inputs or outputs, or when properties needed to refer to specific nodes within the ONNX network. To support networks with multiple inputs or outputs, 
and to provide more expressive tensor declarations, VNN-LIB 2.0 introduces explicit declarations for input and output variables.

An input variable is declared using the \texttt{declare-input} keyword, followed by a variable name, its element type (e.g., \texttt{Real}, \texttt{int8}), and a space-seperated list of integers 
representing the shape of the tensor. Similarly, an output variable uses the \texttt{declare-output} keyword.

The LBNF rules are:
\begin{lstlisting} [
	caption=Input and Output Definition Rules, 
	style=lbnf
]
InputDefinition ::= "(declare-input" VariableName ElementType Int* ")" ;
OutputDefinition ::= "(declare-output" VariableName ElementType Int* ")" ;
\end{lstlisting}
For example, \texttt{(declare-input X Real 1 28 28)} declares an input tensor named \texttt{X} of real numbers with shape $1 \times 28 \times 28$. This syntax allows for tensor-level 
declarations directly, improving conciseness compared to VNN-LIB 1.0, where each tensor element was required to be declared individually.

\myremark{By default, the declared variables correspond to the nodes in the associated ONNX model by their order of declaration. You may specify the ONNX node names explicitly
using the \texttt{onnx-node} keyword under the condition that an ONNX name is provided for each variable.}

\subsection{Intermediate (Hidden) Node Declarations}
A significant extension in VNN-LIB 2.0 is the ability to reference internal (hidden) nodes of a neural network. This is essential for specifying properties on intermediate 
layers, verifying composite architectures (e.g., encoder-decoder models), or for observer-controller systems.

An intermediate node is declared using the \texttt{declare-intermediate} keyword. This declaration includes a variable name for use within the VNN-LIB specification, 
its element type, its tensor shape, and crucially, a string identifier that specifies the corresponding node name in the ONNX graph.

The LBNF rule is:
\begin{lstlisting} [
	caption=Hidden Node Rule, 
	style=lbnf
]
IntermediateDefinition ::= "(declare-intermediate" VariableName ElementType Int* "onnx-node" ":" String ")" ;
\end{lstlisting}
For example, \texttt{(declare-intermediate H1 Real 100 onnx-node:``layer3/relu\_out'')} declares an intermediate variable \texttt{H1} corresponding to the 
ONNX tensor named ``layer3/relu\_out''.

\subsection{Assertion Specification}
VNN-LIB 2.0 continues to support first-order logic formulae as in VNNLIB-1.0. Asertions follow an SMT-LIB-like syntax, defined using parenthesised 
\texttt{(assert\ldots)} expressions. An assertion consists of logical and arithmetic operations over one or more elements of the declared tensors.

\paragraph{Matrix Notation}
Let $X \in I$ be an $n$-dimensional tensor in some generic input domain $I = I^{d_1 \times \cdots \times d_n}$. The ``matrix notation'' represents a specific 
element $x_{i_1, i_2, \dots, i_n}$ of the tensor $X$ as \texttt{X\_$i_1$-$i_2$-\dots-$i_n$}, where $i_1, \dots, i_n$ are the indices of the element of interest in the 
dimensions $d_1, \dots, d_n$. To better clarify, if we consider the 1-D tensor $X \in I^n$, the 2-D tensor $Y \in I^{n \times m}$, and the 3-D tensor 
$Z \in I^{n \times m \times p}$, we will have the following representations:
\begin{itemize}
    \item \texttt{X\_0}, \texttt{X\_1}, \dots, \texttt{X\_$i$}, \dots, \texttt{X\_$n$};
    \item \texttt{Y\_0--0}, \texttt{Y\_0--1}, \dots, \texttt{Y\_$i$-$j$}, \dots, \texttt{Y\_$n$-$m$};
    \item \texttt{Z\_0--0--0}, \texttt{Z\_0--0--1}, \dots, \texttt{Z\_$i$-$j$-$k$}, \dots, \texttt{Z\_$n$-$m$-$p$};
\end{itemize}
In such a representation, \texttt{Z\_$i$-$j$-$k$} corresponds to the element $z_{i,j,k}$ of the tensor $Z$. 

\paragraph{Assertion Example}
For example, a simple Assertion might assert that for a given range of the input neuron $A_1$, the output neuron $B_0$ 
is greater than another output neuron $B_1$:
\begin{lstlisting} [
	caption=Assertion Example, 
	style=lbnf
]
(assert (and (and (>= A_0 0.0) (<= B_0 1.0)) (> B_0 B_1)))}
\end{lstlisting}
More complex properties, including those referencing multiple networks or intermediate nodes, can be constructed using these foundational elements.
