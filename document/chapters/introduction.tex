\chapter{Introduction}
\label{sec:intro}

As neural networks become increasingly integral to safety-critical and high societal impact applications\cite{1, 2, 3}, 
the need for robust verification of their properties is paramount. While neural networks have shown exceptional performance, 
they are susceptible to issues like adversarial examples\cite{3} and discriminatory behaviour\cite{4}.

Since 2023, the VNN-LIB standard\cite{5} has served as a de facto specification language for queries in the neural network verification 
community, notably supporting the annual VNN-COMP\cite{7}. It was conceived to facilitate the standardisation of solver interfaces and 
enable the collection of benchmarks in a common format. 

This document is structured as follows:
\begin{itemize}
\item Chapter~\ref{sec:model} discusses how neural networks models should be represented using the ONNX format.
\item Chapter~\ref{sec:specification_language} details the VNN-LIB query language, including its syntax, scoping rules, typing rules and formal semantics.
\item Chapter~\ref{sec:query_categories} outlines different characteristics of queries, namely linearity, 
reachability, and normality.
\item  Chapter~\ref{sec:solver_interface} introduces a standard command line interface for verifiers.
\item Finally, Chapter~\ref{sec:examples} 
presents examples illustrating how to use VNN-LIB to specify constraints.
\end{itemize}
