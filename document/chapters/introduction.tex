\chapter{Introduction}
\label{sec:intro}

\section{Motivation}

While neural networks have shown exceptional performance across a range of tasks, 
they remain vulnerable to issues such as adversarial examples~\cite{szegedy2013intriguing} and discriminatory behaviour~\cite{buolamwini2018gender}. As these models are deployed in safety-critical and high-impact societal applications, the need for strong guarantees about their behaviour is increasingly rapidly.

As with many formal verification problems, specifications of neural network behaviour can often be reduced to a set of satisfiability queries. These queries are answerable by domain specific solvers which will be referred to as \emph{neural network solvers} or simply \emph{solvers} in this document. 
The \vnnlib{} standard was inspired by the success of SMT-LIB, which provides a unified format for queries to SMT solvers. 
Since its original introduction in 2023~\cite{demarchi2023supporting}, the \vnnlib{} standard has been the de-facto specification language for queries in the neural network verification community, most notably used in the annual VNN-COMP~\cite{brix2023first, bak2021second, muller2022third, brix2023fourth, brix2024fifth} competition. 
The goal of the standard is to facilitate a common solver interface so that solvers can be more easily compared and constrasted and so that higher-level tools can swap between solvers interchangably. 

\section{Document structure}

This document assumes some basic familiarity with neural networks, first-order logic and simple type theory. The document is orgnaised as follows:
\begin{enumerate}
\item \textbf{Chapter~\ref{sec:models} - \nameref{sec:models}}: A high-level description of the pre-existing ONNX standard for representing trained neural network models, and the definition of an abstract interface to ONNX against which the rest of the document is defined.
\item \textbf{Chapter~\ref{sec:specification_language} - \nameref{sec:specification_language}}: A detailed description of the \vnnlib{} language for representing a satisfiability query over one or more neural network models, including syntax, typing and semantics.
\item \textbf{Chapter~\ref{sec:theories-logics} - \nameref{sec:theories-logics}}: A system for describing the different subsets of the query language that a given solver may support.
\item \textbf{Chapter~\ref{sec:solver_interface} - \nameref{sec:solver_interface}}: A standardised interface for allowing users to invoke and query the capabilities of a solver.
\end{enumerate}

\section{Document versioning}

This document uses semantic versioning of the form \texttt{major.minor.patch}. Patch versions will alter the wording of the specification document, minor versions will include conservative extensions to the specification, while changes to the major version number will result from major backwards-incompatible changes.

\subsection*{Changelog for v2.0.0}

\paragraph{Authors:} Allen Antony, Ann Roy, Matthew Daggitt with input from Stefano Demarchi, Andrea Gimelli.

\noindent \paragraph{Acknowledgements:} Version 2.0.0 of the standard was developed with the help of many others in the neural network verification community.
Particular thanks must given to the following people, who provided many helpful suggestions, constructive criticism and encouragement: Taylor Johnson, Samuel Teuber, Wen Kokke, Julien Girard, Guilhem Ardouin, Augustin Lemesle, Michele Alberti, Julien Lehmann, Thomas Flinkow, Edoardo Manino, Guy Amir, Omri Isac, Guy Katz, Idan Refaeli, David Shriver and Christopher Brix.

\noindent \paragraph{Changes:}
\begin{itemize}
\item Added the concept of a \emph{network theory}, thereby explicitly relativising \vnnlib{} against ONNX.
\item Added a formal grammar for query language.
\item Added explicit network declarations to the query language.
\item Added support for networks with multiple inputs/outputs and hidden layers.
\item Added support for multiple networks.
\item Added a formal type system and semantics.
\item Added theories and logics.
\item Introduction of the command-line interface: \texttt{verify} and \texttt{supports}.
\end{itemize}

\subsection*{Changelog for v1.0}

\paragraph{Authors:} Stefano Demarchi, Dario Guidotti, Luca Pulina, Armando Tacchella

\paragraph{Contents:}
\begin{itemize}
\item Initial release.
\item Outline of goal.
\item Proposal of initial syntax of the query language with examples.
\end{itemize}
