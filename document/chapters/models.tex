\chapter{Network Models}
\label{sec:models}
%
At a high level, networks can be seen as functions 
$\nu : I^n \to O^m$, mapping one or more $n$-dimensional \emph{input domains}
$I^n$ ($n > 0$) to one or more $m$-dimensional \emph{output domains} $O^m$ ($m>0$). 
We argue that this representation captures most cases of practical
interest.

For instance, a network computing an approximation
of some function $f: \mathbb{R}^n \to \mathbb{R}$ would have $I = O =
\mathbb{R}$, whereas a network classifying 8-bit images of size $h \times v$ in
two classes would be defined as ${\nu: \{0,\ldots,255\}^{h \cdot v}
\to \{0, 1\}}$ with $I=\{0, \ldots, 255\}$ and $O = \{0,1\}$.

At a lower level, neural networks can be described as a computational graph, specifically a directed acyclic graph (DAG), where nodes represent operations 
(e.g., addition, multiplication, activation functions) and edges represent the flow of data (tensors) between these operations. The DAG receives (one or more) input tensors, 
processes them through a series of operations, and produces (one or more) output tensors.

\section{Network Formats}
\label{sec:model_formats}
Neural networks are represented in serialized formats that allow for model saving, sharing, and deployment. Framework-native formats (e.g., TensorFlow's .ckpt, PyTorch's .pth) are 
designed for integration with a single deployment environment, while interoperable formats are developed to facilitate cross-framework compatibility. 

\subsection{Interoperable Network Formats}
Interoperable formats are designed to allow neural networks to be shared and executed across different frameworks and hardware platforms. Some of the most notable formats include:
\begin{itemize}
	\item \textbf{ONNX (Open Neural Network Exchange)}: A widely adopted format that supports a variety of neural network architectures and operations, enabling models to be shared across different frameworks.
	\item \textbf{NNEF (Neural Network Exchange Format)}: A format developed by the Khronos Group, designed for efficient deployment of neural networks across different hardware platforms.
	\item \textbf{Safetensors}: A format that ensures the security and integrity of neural network models, preventing tampering and ensuring safe loading.
	\item \textbf{GGUF}: A format originating from the Llama project, which is designed for efficient execution of quantised models on low-power hardware. 
\end{itemize}

\subsection{Justification for ONNX}
ONNX is chosen as the standard format for \vnnlib{} for several reasons:
\begin{itemize}
	\item \textbf{Widespread Industry Adoption}: ONNX conversion is supported by frameworks such as PyTorch, and Tensorflow, making it ideal for framework-agnostic model representation.
	\item \textbf{Rich Operator Set}: ONNX supports a wide range of operators, enabling the representation of complex neural network architectures.
	\item \textbf{Neural Network Verification Adoption}: ONNX has an established presence in NNV research, since VNN-COMP has used ONNX as the standard format for benchmarks.
	\item \textbf{Rigorous Specification}: ONNX models are rigorously specified computation graphs with strongly-typed data tensors and versioned (via opsets) operators, eliminating ambiguity.
\end{itemize}

\section{ONNX Format Overview}
\label{sec:onnx_overview}
ONNX is a standardized format for representing neural networks as directed acyclic graphs (DAGs). An ONNX model is serialized into a single binary file (with a .onnx extension) using Google's Protocol Buffers 
(Protobuf), a language and platform-neutral mechanism for serializing structured data

\subsection{ONNX Model Structure}
The ONNX model \texttt{GraphProto} contains the following key components:
\begin{itemize}
	\item \textbf{Node}: Each node has an operation (e.g., convolution, activation), a type (e.g., Conv, Relu) and attributes 
	(e.g., kernel size, stride), and contains a list of inputs and outputs, representing the tensors that flow into and out of the operation.
	\item \textbf{Initializer}: These are tensors that are used as constants in the model, such as weights and biases.
	\item \textbf{Metadata}: This describes attributes of the model such as model version, author, and description.
	\item \textbf{Input/Output}: These specify the external inputs and outputs of the model, including their names, datatypes, and shapes.
\end{itemize}

\subsection{ONNX Tensors}
Data within ONNX models is strongly typed, with each tensor having a specific numeric type and shape. The supported properties for a tensor include:
\begin{itemize}
	\item \textbf{Element Type}: ONNX defines a standard set of numeric types \href{https://onnx.ai/onnx/intro/concepts.html#element-type}{here}. 
	The most commonly used types for neural network verification are: 32-bit floating point (FLOAT), and 16-bit floating point (FLOAT16). Signed and unsigned integers (e.g., INT32, UINT8) are also supported, 
	but less common in neural networks.
	\item \textbf{Shape}: The shape of a tensor is defined as a list of integers, where each integer represents the size of the corresponding dimension. For example, a tensor with shape [3, 224, 224] 
	represents an image with 3 color channels (RGB) and dimensions 224x224 pixels.
	\item \textbf{Values}: A contiguous block of memory containing data of a specific numeric type.
\end{itemize}

\subsection{ONNX Opsets}
ONNX uses a versioning system called \emph{opsets} to manage the evolution of operators. Each operator is associated with a specific opset version, which defines its semantics and attributes.
This ensures that the precise mathematical definition and behaviour of each operator is well-defined and can evolve over time without breaking existing models. The ONNX model file declares the opset
version used.
	
\section{\vnnlib{} Support for ONNX}
\label{sec:onnx_support}
\vnnlib{} supports query specification for any abritary ONNX model that is sequential. In practice however, most verifiers support a subset of ONNX operators and datatypes. Additionally, verifiers may not support
models with multiple inputs and/or outputs. To ensure clarity, transparency, and interoperability, \vnnlib{} defines as part of its verifier interface (Section~\ref{sec:solver_interface}), a capability query 
mechanism that allows users to determine the ONNX operators, datatypes, and model structures supported by a specific verifier. This is detailed in Section~\ref{sec:global_capabilities}.

\section{ONNX Operators}
\label{sec:supported_operators}
The following operators cover almost every benchmark provided in the
VNN-COMP repositories for sequential networks; other kinds of networks
(ResNet, Recurrent, etc.) are often based on ``exotic'' and, in general,
peculiar operators that do not lie in this list.

\begin{itemize}
	\item \emph{Add (Add)} operator performs the element-wise sum of
		a tensor and a scalar. We strongly encourage to use the 
		\textit{Gemm} operator when paired with \textit{MatMul}.
	
	\item \emph{AveragePool (Average Pooling)} operator
	  supports downsampling with averaging.
	
	\item \emph{BatchNormalization (Batch
	  Normalization)} operator supports adjusting and scaling the
	  activations functions, and it is expressive enough to represent
	  general batch normalization.
	  
	\item \emph{Concat (Concatenation)} operator concatenates a list
		of tensors into a single tensor, with the same shape except for
		the axis to concatenate on.	
	
	\item \emph{Conv (Convolutional)} operator supports
	  all the attributes to encode a generic convolutional layer.
	
	\item \emph{Dropout (Dropout)} operator supports
	  random dropping of units (during training). This operator should not
	  appear on trained models.  
	  
	\item \emph{Flatten (Flatten)} this operator converts multidimensional
	  arrays (tensors) to single dimensional ones; it is used instead of
	  \emph{Reshape} in some of the models in the zoo.
	  
	\item \emph{Gemm (General Matrix Multiplication)}
	  operator encodes matrix multiplication possibly with a scalar
	  coefficient and the addition of another matrix; as such \emph{Gemm}
	  can encode fully connected layers in neural networks.
	
	\item \emph{LRN (Local Response Normalization)}
	  operator supports normalization over local input regions; it is uatilized
	  in Alexnet and derived networks.
	  
	\item \emph{MatMul (Matrix Multiplication)} operator performs a
		numPy-like matrix multiplication. We strongly encourage to use
		the \textit{Gemm} operator.
	  
	\item \emph{MaxPool (Maximum pooling)} operator supports
	  downsampling with maximization.
	
	\item \emph{ReLU (Rectified Linear Unit)} operator
	  encodes the corresponding activation function $\sigma(x) = \max(0, x)$.
	  
	\item \emph{Reshape (Reshape)} operator supports
	  reshaping of the tensor's dimensions.
	  
	\item \emph{Sigmoid (Logistic Unit)} operator
	  encodes the corresponding activation function $\sigma(x) =
	  \frac{1}{1 + e^{-x}}$.
	  
	\item \emph{SoftMax (Softmax Unit)} operator transforms
	  vectors into probabilities, e.g., for selecting among different
	  classes and it is commonly utilized in state of the art
	  networks.
	  
	\item \emph{Sub (Sub)} operator performs element-wise binary subtraction
		between two tensors.
	
	\item \emph{Unsqueeze (Unsqueeze)} operator removes dimensions of size
	  1 from tensors, and it is utilized, e.g., in \emph{Densenet} and
	  \emph{Inception2}.
\end{itemize}



