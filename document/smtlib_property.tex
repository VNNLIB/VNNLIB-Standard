Inputs and outputs of operators are \emph{tensors}, i.e.,
multidimensional arrays over some domain, usually numerical. 
If we let $\mathbb{D}$ be any such domain, a $k$-dimensional 
tensor on $\mathbb{D}$ is denoted as $x \in \mathbb{D}^{n_1 
	\times \ldots \times n_k}$.
For example, a vector of $n$ real numbers is a 1-dimensional
tensor $x \in \mathbb{R}^n$, whereas a matrix of $n \times n$ 
Booleans is a 2-dimensional tensor $x \in \mathbb{B}^{n 
	\times n}$ with $\mathbb{B} = \{0, 1\}$. A specific element 
of a tensor can be singled-out via \emph{subscripting}. 

Given a $k$-dimensional tensor $x \in \mathbb{D}^{n_1 \times 
	\ldots \times n_k}$, the element $x_{i_1, \ldots, i_k} \in 
	\mathbb{D}$ is a scalar corresponding to the indexes 
${i_1, \ldots, i_k}$. For example, in a vector of real numbers 
$x \in \mathbb{R}^n$, $x_1$ is the first element, $x_2$ the second 
and so on. In a matrix of Boleans $x \in \mathbb{B}^{n \times
  n}$, $x_{1,1}$ is the first element of the first row, $x_{2,1}$ 
is the first element of the second and so on.

An \emph{operator} $f$ is a function on tensors 
$f: \mathbb{D}^{n_{1} \times n_h} \to \mathbb{D}^{m_{1} \times m_k}$
where $h$ is the dimension of the input tensor and $k$ is the 
dimension of the output tensor. Given a set $F = \{f_1, \ldots, 
	f_p\}$ of $p$ operators, a \emph{feedforward neural network}
is a function $\nu = f_p(f_{p-1}(\ldots f_2(f_1(x))\ldots))$ obtained
through the composition of the operators in $F$ assuming that the 
dimensions of their inputs and outputs are \emph{compatible}, i.e.,
if the  output of $f_i$ is a $k$-dimensional tensor, then the input
of $f_{i+1}$ is also a $k$-dimensional tensor, for all $1 \leq i < p$.

Given a neural network $\nu : \mathbb{D}^{n_{1} \times n_h} \to
\mathbb{D}^{m_{1} \times m_k}$ built on the set of operators $\{f_1,
\ldots, f_p\}$, let $x \in \mathbb{D}^{n_{1} \times n_h}$ denote
the input of $\nu$ and $y_1, \ldots, y_p$ denote the outputs of the
operators $f_1, \ldots, f_p$ --- therefore $y_p$ is also the output
$y$ of $\nu$. We assume that, in general, a \emph{property} is a first
order formula $P(x, y_1, \ldots y_p)$ which should be satisfied given 
$\nu$. More formally, given $p$ bounded sets $X_1, \ldots, X_p$ in $I$ 
such that $\Pi = \bigcup_{i=1}^p X_i$ and $s$ bounded sets $Y_1, 
\ldots, Y_s$ in $O$ such that $\Sigma = \bigcup_{i=1}^s Y_i$, we wish
to prove that  
\begin{equation}
	\label{eq:verif}
	\forall x \in \Pi \rightarrow \nu(x) \in \Sigma.
\end{equation}
The definition of the property given in equation (\ref{eq:verif})
consists of a \textit{pre-}condition $x \in \Pi$ and a 
\textit{post-}condition $\nu(x) \in \Sigma$. The 
\textit{pre-}condition encodes the bounds of the input space, i.e.,
bounds the variables that are fed to the network, and the 
\textit{post-}condition defines the safe zone, outside which the 
verification task fails.

The SMT-LIB language is a well-known language used to formalize 
Satisfiability Modulo Theories problems, and is expressive enough to
represent the verification properties of interest. In this language, 
it is possible to define both the \textit{pre-} and 
\textit{post-}conditions at once, by defining the variables for the
input and the output of the neural network. In the following we
show some examples of networks and corresponding properties in the
SMT-LIB language.

\myremark{Note that the input and output variable names should match
	the identifiers of the input and of the last node in the network.}